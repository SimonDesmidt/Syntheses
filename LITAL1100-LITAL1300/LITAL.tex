\documentclass[12pt, openany]{report}
\usepackage[utf8]{inputenc}
\usepackage[T1]{fontenc}
\usepackage{amsmath,amsfonts,amssymb}
\usepackage{amssymb}
\usepackage{multicol}
\usepackage[a4paper,left=2.5cm,right=2.5cm,top=2.5cm,bottom=2.5cm]{geometry}
\usepackage[french]{babel}
\usepackage{libertine}
\usepackage{graphicx}
\usepackage{wrapfig}
\usepackage{float}
\usepackage{enumitem}
\usepackage[]{titletoc}
\usepackage{empheq}
\usepackage{titlesec}
\usepackage{textcomp}
\usepackage{caption}
\usepackage{tabularray}
\usepackage{subcaption}
\usepackage[bottom]{footmisc}
\usepackage{pdfpages}
\usepackage{tabularx}
\usepackage{amsthm}
\usepackage[skins]{tcolorbox}
\titleformat{\chapter}[display]
  {\normalfont\bfseries}{}{0pt}{\Huge}
\usepackage{hyperref}
\newcommand{\HRule}{\rule{\linewidth}{0.5mm}}
\usepackage{silence}
\WarningFilter{latex}{Overfull \hbox}

\begin{document}

\begin{titlepage}
    \begin{sffamily}
    \begin{center}
        \includegraphics[scale=0.25]{img/page_de_garde.png} \\[1cm]
        \HRule \\[0.4cm]
        { \huge \bfseries LITAL1100/1300 Italien - niveaux élémentaire et moyen \\[0.4cm] }
    
        \HRule \\[1.5cm]
        \textsc{\LARGE Simon Desmidt\\ Issambre L'Hermite Dumont}\\[1cm]
        \vfill
        \vspace{2cm}
        {\large Années académiques 2024-2025-2026}
        \vspace{0.4cm}
         
        \includegraphics[width=0.15\textwidth]{img/ilv.png}
        
        UCLouvain\\
    
    \end{center}
    \end{sffamily}
\end{titlepage}

\setcounter{tocdepth}{1}
\tableofcontents
\chapter{Conjugaison}
\section{Indicatif présent}
\subsection{Essere/Avere}
\begin{center}
    \begin{tabular}{c|c|c}
        & ESSERE & AVERE \\ \hline
        io & sono & ho\\
        tu & sei & hai\\
        lui/lei/Lei & è & ha\\ 
        noi & siamo & abbiamo \\
        voi & siete & avete \\
        loro & sono & hanno\\
    \end{tabular}
\end{center}
\subsection{Règle générale}
\begin{minipage}{.49\textwidth}
    \vspace{.75cm}
    \begin{center}
        \begin{tabular}{c||c|c|c}
            & ARE & ERE & IRE\\
            \hline
            io & -o & -o & -o\\
            tu & -i & -i & -i\\
            lui/lei/Lei & -a & -e & -e\\
            noi & -iamo & -iamo & -iamo\\
            voi & -ate & -ete & -ite\\
            loro & -ano & -ono & -ono\\
        \end{tabular}
    \end{center}
\end{minipage}
\begin{minipage}{.5\textwidth}
    \centering 
    \begin{figure}[H]
        \includegraphics[width = \textwidth]{img/irregolari_presente.png}
    \end{figure}
\end{minipage}\\
\begin{minipage}{.35\textwidth}
    Cas particuliers de spedire (envoyer), preferire (préférer), finire (finir) et capire (comprendre):
    \begin{center}
        \begin{tabular}{c||c}
            io & -isco \\
            tu & -isci \\
            lui & -isce \\
            noi & -iamo \\
            voi & -ite \\
            loro & -iscono \\
        \end{tabular}
    \end{center}
\end{minipage}
\begin{minipage}{.65\textwidth}
    \centering 
    \begin{figure}[H]
        \includegraphics[width = \textwidth]{img/isco.png}
    \end{figure}
\end{minipage}
\subsection{Cas particuliers}
\begin{minipage}{.315\textwidth}
    \begin{center}
        \begin{tabular}{c||c}
            & FARE\\ \hline
            io & faccio\\
            tu & fai\\
            lui & fa\\
            noi & facciamo\\
            voi & fate\\
            loro & fanno\\
        \end{tabular}
    \end{center}
\end{minipage}
\begin{minipage}{.315\textwidth}
    \begin{center}
        \begin{tabular}{c||c}
            & ANDARE\\ \hline
            io & vado\\
            tu & vai\\
            lui & va\\
            noi & andiamo\\
            voi & andate\\
            loro & vanno\\
        \end{tabular}
    \end{center}
\end{minipage}
\begin{minipage}{.315\textwidth}
    \begin{center}
        \begin{tabular}{c||c}
            & VENIRE\\ \hline
            io & vengo\\
            tu & vieni\\
            lui & viene\\
            noi & veniamo\\
            voi & venite\\
            loro & vengono\\
        \end{tabular}
    \end{center}
\end{minipage}
\begin{minipage}{.33\textwidth}
    \begin{center}
        \begin{tabular}{c||c}
            & STARE\\ \hline
            io & sto\\
            tu & stai\\
            lui & sta\\
            noi & stiamo\\
            voi & state\\
            loro & stanno\\
        \end{tabular}
    \end{center}
\end{minipage}
\begin{minipage}{.33\textwidth}
    \begin{center}
        \begin{tabular}{c||c}
            & USCIRE (sortir)\\ \hline
            io & esco\\
            tu & esci\\
            lui & esce\\
            noi & usciamo\\
            voi & uscite\\
            loro & escono\\
        \end{tabular}
    \end{center}
\end{minipage}
\begin{minipage}{.33\textwidth}
    \begin{center}
        \begin{tabular}{c||c}
            & DIRE\\ \hline
            io & dico\\
            tu & dici\\
            lui & dice\\
            noi & diciamo\\
            voi & dicite\\
            loro & dicono\\
        \end{tabular}
    \end{center}
\end{minipage}
\subsection{Verbes réfléchis}
Les verbes réfléchis fonctionnent comme les réguliers, avec les pronoms réfléchis en plus:
\begin{center}
    \begin{tabular}{c||c|c|c|c}
        & & ARE & ERE & IRE\\
        \hline
        io & mi & -o & -o & -o\\
        tu & ti & -i & -i & -i\\
        lui/lei/Lei & si & -a & -e & -e\\
        noi & ci & -iamo & -iamo & -iamo\\
        voi & vi & -ate & -ete & -ite\\
        loro & si &-ano & -ono & -ono\\
    \end{tabular}
\end{center}
\subsection{Verbes modaux}
\begin{minipage}{.315\textwidth}
    \begin{center}
        \begin{tabular}{c||c}
            & VOLERE \\ \hline
            io & voglio\\
            tu & vuoi\\
            lui & vuole\\
            noi & vogliamo\\
            voi & volete \\
            loro & vogliono \\
        \end{tabular}
    \end{center}
\end{minipage}
\begin{minipage}{.315\textwidth}
    \begin{center}
        \begin{tabular}{c||c}
            & POTERE \\ \hline
            io & posso \\
            tu & puoi \\
            lui & può\\
            noi & possiamo\\
            voi & potete\\
            loro & possono\\
        \end{tabular}
    \end{center}
\end{minipage}
\begin{minipage}{.315\textwidth}
    \begin{center}
        \begin{tabular}{c||c}
            & DOVERE \\ \hline
            io & devo/debbo\\
            tu & devi \\
            lui & deve \\
            noi & dobbiamo \\
            voi & dovete \\
            loro & devono/debbono\\
        \end{tabular}
    \end{center}
\end{minipage}
\section{Gérondif}
\begin{center}
    \begin{tabular}{c|c|c}
        ARE & ERE & IRE\\ \hline 
        -ando & -endo & -endo\\
    \end{tabular}
\end{center}
Le gérondif sert à former "Stare + gérondif", i.e. "être en train de".
\section{Impératif}
\begin{center}
    \begin{tabular}{c|c|c|c}
        & ARE & ERE & IRE\\ \hline
        tu & -a & -i & -i \\
        Lei & -i & -a & -a\\
        noi & -iamo & -iamo & -iamo \\
        voi & -ate & -ete & -ite \\
    \end{tabular}
\end{center}
\begin{itemize}
    \item [$\rightarrow$] Remarque : la forme polie prend la forme de la première personne du singulier à l'indicatif et modifie la terminaison (important pour les irréguliers, e.g. venire $\rightarrow$ venga!).
\end{itemize}
La forme négative est la même pour noi et voi. Elle est "non + infinitif" au tu. 
\subsection{Cas particuliers au TU}
\begin{center}
    \begin{tabular}{c|c}
        vai & va'\\
        fai & fa'\\
        stai & sta'\\
        dai (dare = donner) & da'\\
        dici (dire = dire) & di'\\
        essere & sii/cerca di essere (à l'oral)\\
        avere & abbi\\ 
    \end{tabular}
\end{center}
Quand l'impératif est utilisé avec un pronom (e.g. change-la!), on fusionne le verbe et le pronom, e.g. compraLA, leggiamoLO, apriteLE. De plus, les verbes en une syllabe doublent la consonne : mi dire devient dimmi.
\subsection{Cas particuliers au LEI}
\begin{center}
    \begin{tabular}{c|c|c}
        Infinitivo & TU & LEI \\ \hline 
        Stare & sta' & stia\\
        Dare & da' & dia \\
        Dire & di' & dica \\
        Fare & fa' & faccia\\
        Andare & va' & vada \\
        Essere & sii & sia \\
        Avere & abbi & abbia 
    \end{tabular}
\end{center}
\section{Imparfait}
\begin{center}
    \begin{tabular}{c|c|c|c}
        & ARE & ERE & IRE\\ \hline
        io & -avo & -evo & -ivo\\
        tu & -avi & -evi & -ivi\\
        lui & -ava & -eva & -iva\\
        noi & -avamo & -evamo & -ivamo\\
        voi & -avate & -evate & -ivate\\
        loro & -avano & -evano & -ivano\\
    \end{tabular}
\end{center}
\begin{itemize}
    \item [$\rightarrow$] Remarque : tous les irréguliers deviennent réguliers, sauf cas très particuliers.
\end{itemize}
\subsection{Cas particuliers}
\begin{minipage}{.315\textwidth}
    \begin{center}
        \begin{tabular}{c||c}
            & ESSERE \\ \hline
            io & ero \\
            tu & eri \\
            lui & era \\
            noi & eravamo \\
            voi & eravate \\
            loro & erano \\
        \end{tabular}
    \end{center}
\end{minipage}
\begin{minipage}{.315\textwidth}
    \begin{center}
        \begin{tabular}{c||c}
            & FARE \\ \hline
            io & facevo \\
            tu & facevi \\
            lui & faceva \\
            noi & facevamo\\
            voi & facevate\\
            loro & facevano \\
        \end{tabular}
    \end{center}
\end{minipage}
\begin{minipage}{.315\textwidth}
    \begin{center}
        \begin{tabular}{c||c}
            & DIRE \\ \hline
            io & dicevo \\
            tu & dicevi \\
            lui & diceva \\
            noi & dicevamo\\
            voi & dicevate \\
            loro & dicevano\\
        \end{tabular}
    \end{center}
\end{minipage}
\section{Forme impersonnelle}
En italien, il existe une forme impersonnelle comme le "on" en français, e.g. on mange bien en Italie. Le sujet dans la phrase est "si" dans tous les cas sauf les verbes réfléchis, où le sujet est "ci". Par exemple,
\begin{itemize}
    \item Si on n'étudie pas, on ne réussit pas $\rightarrow$ Se non \textcolor{red}{si} studia, non \textcolor{red}{si} impara.
    \item On se lève tôt $\rightarrow$ \textcolor{red}{Ci} si svelgia presto.
\end{itemize}
\section{Forme progressive}
Il existe une forme progressive en conjugaison italienne. Elle se forme comme ceci: 
\begin{center}
    STARE (présent, imparfait, etc) + GERONDIF
\end{center}
Cette forme existe avec tous les temps et sert à décrire des actions en cours, quel que soit le temps  
\section{Passé composé}
Le passé composé se forme comme en français : auxiliaire (essere/avere) + participe passé.
\subsection{Participe passé}
\begin{center}
    \begin{tabular}{c|c|c}
        ARE & ERE & IRE\\ \hline
        -ato & -uto & -ito\\
    \end{tabular}
\end{center}
Avec l'auxiliaire essere, on fait toujours l'accord entre le sujet et le participe passé. Avec avere, on accorde uniquement avec les COD pronominalisés. 
\begin{itemize}
    \item [$\to$] Remarque : la majorité des verbes -ERE sont irréguliers. 
    \item [$\to$] Remarque : on met les pronoms compléments avant l'auxiliaire dans la phrase. 
    \item [$\to$] Remarque : l'accent est toujours sur l'antépénultième syllabe, i.e. le a/u/i de la terminaison. 
\end{itemize}
\subsection{Quel auxiliaire?}
On utilise "essere" pour :
\begin{itemize}
    \item les verbes de mouvement, les verbes d'état et de lieu et les verbes réfléchis;
\end{itemize}
On utilise "avere" pour :
\begin{itemize}
    \item Les verbes transitifs et quelques verbes intransitifs (dormire, ridere, piangere, correre, camminare, lavorare,etc.).
\end{itemize} 
Certains verbes peuvent prendre les deux auxiliaires, dépendant du contexte et de la formation de la phrase, e.g. cambiare, passare, finire, etc. 
\subsection{Exceptions}
\begin{itemize}
    \item Essere : io sono stato (je suis été et pas j'ai été).
    \item Les verbes qui indiquent le temps utilisent l'auxiliaire essere : il film è durato due ore.
    \item Piacere utilise avere : questo libro mi è piaciuto.
    \item Les verbes de la météo utilisent essere : ieri è piovuto.
\end{itemize}
\section{Plus-que-parfait}
L'utilisation du plus-que-parfait en italien est la même qu'en français. Il se forme avec l'auxiliaire (essere/avere) à l'imparfait + participe passé.
\begin{center}
    ERO / AVEVO + PARTICIPE PASSÉ
\end{center}
\section{Futur simple}
\subsection{Règle générale}
\begin{center}
    \begin{tabular}{c|c|c|c}
        & ARE & ERE & IRE\\ \hline
        io & -erò & -erò & -irò\\
        tu & -erai & -erai & -irai\\
        lui & -erà & -erà & -irà\\
        noi & -eremo & -eremo & -iremo\\
        voi & -erete & -erete & -irete\\
        loro & -eranno & -eranno & -iranno\\
    \end{tabular}
\end{center} 
\subsection{Cas particuliers}
\begin{minipage}{.478\textwidth}
    \begin{center}
        \begin{tabular}{c|c|c}
            Français & Infinitif & Première personne \\ \hline 
            Être & Essere & sarò\\ 
            Avoir & Avere & avrò\\
            Aller & Andare & andrò\\
            Devoir & Dovere & dovrò\\
            Pouvoir & Potere & potrò\\
            Savoir & Sapere & saprò\\
            Voir & Vedere & vedrò\\
            Donner & Dare & darò\\
            &&\\
        \end{tabular}
    \end{center}
\end{minipage}
\begin{minipage}{.478\textwidth}
    \begin{center}
        \begin{tabular}{c|c|c}
            Français & Infinitif & Première personne \\ \hline
            Vivre & Vivere & vivrò\\
            Boire & Bere & berrò\\
            Venir & Venire & verrò\\ 
            Vouloir & Volere & vorrò\\
            Dire & Dire & dirò\\
            Faire & Fare & farò\\
            Être & Stare & starò\\
            Rester & Rimanere & rimarrò \\
            Tenir & Tenere & terrò\\
        \end{tabular}
    \end{center}
\end{minipage}
\begin{itemize}
    \item [$\to$] Remarque : les verbes en -care et -gare ont un h (gherò, cherò).
\end{itemize}
\subsection{Expressions de temporalité}
\begin{center}
    \begin{tabular}{c|c|c|c}
        \multicolumn{2}{c|}{\textbf{Futur}} & \multicolumn{2}{c}{\textbf{Passé}} \\ \hline
        Domani & Demain & Ieri & Hier \\
        Dopodomani & Après-demain & L'altroieri & Avant-hier \\
        Fra/tra due giorni & Dans deux jours & Due giorni fa & Il y a deux jours \\
        La settimana prossima & La semaine prochaine & La settimana scorsa & La semaine passée \\
        Domani mattina & Demain matin & Ieri mattina & Hier matin \\
        Domani sera/notte & Demain soir & Ieri sera & Hier soir \\
    \end{tabular}
\end{center}
\section{Futur antérieur}
\begin{center}
    Essere/Avere (futur simple) + Participe passé 
\end{center}
\begin{itemize}
    \item [$\to$] Remarque : l'utilisation est la même qu'en français.
\end{itemize}
\section{Random}
\begin{center}
    Avere + appena + fatto + infinitif 
\end{center}
Cette formulation signifie "venir de faire", et avere peut se conjuguer à tous les temps.
\section{Conditionnel présent}
\begin{center}
    \begin{tabular}{c|c|c|c}
        & ARE & ERE & IRE\\ \hline
        io & -erei & -erei & -irei\\
        tu & -eresti & -eresti & -iresti\\
        lui & -erebbe & -erebbe & -irebbe\\
        noi & -eremmo & -eremmo & -iremmo\\
        voi & -ereste & -ereste & -ireste\\
        loro & -erebbero & -erebbero & -irebbero\\
    \end{tabular}
\end{center} 
Son utilisation est la même qu'en français. 
\subsection{Irréguliers}
\begin{center}
    \begin{tabular}{c|c||c|c||c|c}
        Infinitif & Forme irrégulière & Infinitif & Forme irrégulière & Infinitif & Forme irrégulière\\ \hline 
        Essere & sarei & Avere & avrei & Bere & berrei\\
        Dare & darei & Andare & andrei & Rimanere & rimarrei\\
        Dire & direi & Dovere & dovrei & Venire & verrei\\
        Fare & farei & Potere & potrei & Volere & vorrei\\
        Stare & starei & Sapere & saprei & Tenere & terrei\\
        Avere & avrei & Vedere & vedrei & Tradurre & tradurrei\\
    \end{tabular}
\end{center}
\section{Conditionnel passé}
\begin{center}
    Avere/Essere (Conditionnel présent) + participe passé
\end{center}
Son utilisation est la même qu'en français. 
\section{Participes passés irréguliers}
\begin{center}
    \begin{tabular}{c|c|c|c}
        Famille & Infinitif & Participe passé & Traduction\\ \hline 
        ggere $\to$ tto & correggere & corretto & corriger \\
        & friggere & fritto & frire\\
        & leggere & letto & lire \\
        & dire & detto & dire\\
        & fare & fatto & faire\\
        & rompere & rotto & casser \\
        & scrivere & scritto & écrire\\ \hline 
        dere $\to$ so/sto & accendere & acceso & allumer\\
        ndere $\to$ so & chiudere & chiuso & fermer \\
        & nascondere & nascosto & cacher\\
        & offendere & offeso & offenser\\
        & sorprendere & sorpreso & surprendre\\
        & ridere & riso & rire\\
        & decidere & deciso & décider\\
        & dividere & diviso & diviser\\
        & deludere & deluso & décevoir \\
        & prendere & preso & prendre\\
        & scendere & sceso & descendre\\
        & spendere & speso & dépenser \\
        & uccidere & ucciso & tuer \\\hline 
        & perdere & perso & perdre \\
        orre $\to$ osto & proporre & proposto & proposer \\
        & chiedere & chiesto & demander\\
        & rimanere & rimasto & rester\\
        & rispondere & risposto & répondre\\
        & vedere & visto & voir\\ \hline 
        rire $\to$ rto & aprire & aperto & ouvrir\\
        & morire & morto & mourir\\
        & offrire & offerto & offrir\\
        & soffrire & sofferto & souffrir\\ \hline 
        tt $\to$ ss & mettere & messo & mettre\\
        & smettere & smesso & arrêter\\
        & permettere & permesso & permettre \\
        & promettere & promesso & promettre \\
        & discutere & discusso & discuter\\
        & muovere & mosso & bouger\\
        & esprimere & espresso & exprimer \\
        & succedere & successo & arriver (happen)\\ \hline 
    \end{tabular}
\end{center}
\begin{center}
    \begin{tabular}{c|c|c|c}
        Famille & Infinitif & Participe passé & Traduction\\ \hline 
        ngere/ncere $\to$ nto & giungere & giunto & arriver\\
        & convincere & convinto & convaincre\\
        & spingere & spinto & pousser\\
        & spegnere & spento & éteindre\\
        & piangere & pianto & pleurer \\
        & vincere & vinto & gagner \\ \hline 
        & piacere & piaciuto & plaire \\
        & conoscere & consciuto & connaître \\
        & nascere & nato & naître\\
        & bere & bevuto & boire \\
        & correre & corso & courir \\
        & essere/stare & stato & être \\
        gliere $\to$ lto & scegliere & scelto & choisir \\
        urre $\to$ otto & tradurre & tradotto & traduire\\
        & venire & venuto & venir \\
        & vivere & vissuto & vivre \\
    \end{tabular}
\end{center}
\chapter{Grammaire}
\section{Formation d'une phrase}
Les structures grammaticales française et italienne sont très similaires. De manière générale, ce qui s'applique à l'une s'applique aussi à l'autre.\\
Une phrase de base, en français comme en italien, est formée de la manière suivante : 
\begin{center}
    \textcolor{red}{Sujet} + \textcolor{blue}{Verbe} + \textcolor{green}{Complément}
\end{center}
Exemple : \textcolor{red}{Marie} \textcolor{blue}{construis} \textcolor{green}{une maison} et en italien : \textcolor{red}{Maria} \textcolor{blue}{costruisce} \textcolor{green}{una casa}.
Toute phrase ne contenant pas ces éléments ne veut rien dire (ou est très basique, comme "je mange").
\subsection{\textcolor{red}{Sujet}}
Le sujet est la personne ou la chose qui fait l'action. Il s'agit en général d'un nom ou d'un pronom avec en option des compléments du nom ou des adjectifs.\\ Exemples : 
\begin{itemize}
    \item Le vieil homme -- il ragazzo alto;
    \item Mon ami -- il mio amico;
    \item Le groupe d'enfants malades -- il gruppo di bambini malati.
\end{itemize}
Si le sujet est un pronom personnel, il est souvent omis en italien, sauf pour insister sur le sujet.\\ Exemples : 
\begin{itemize}
    \item Je mange une pomme -- Mangio una mela;
    \item Elle, elle mange une pomme -- Lei mangia una mela.
\end{itemize}
\subsection{\textcolor{blue}{Verbe}}
Le verbe exprime l'action. Il se conjugue en accord avec le sujet. En cas de verbe composé, e.g. passé composé, le participe passé s'accorde aussi dans certains cas avec le complément (voir plus bas). 
\subsection{\textcolor{green}{Complément du verbe}}
Le complément est la personne ou la chose qui subit l'action. Il est direct ou indirect (COD ou COI). Comme le sujet, il peut s'agir  d'un nom, pronom ou groupe de mots organisés autour d'un nom.\\ 
\begin{itemize}
    \item Un complément est COD quand il répond à la question QUOI/QUI? Exemples : 
    \begin{itemize}
        \item [$\bullet$] Je mange \textcolor{green}{une pomme}. Je mange QUOI? $\rightarrow$ une pomme = COD;
    \end{itemize}
    \item Un complément est COI quand il répond à la question À QUI/À QUOI / DE QUI/DE QUOI? Exemples :
    \begin{itemize}
        \item [$\bullet$] Je parle \textcolor{green}{à Marie}. Je parle À QUI? $\rightarrow$ à Marie = COI;
    \end{itemize}
\end{itemize}
Une phrase ne peut contenir qu'un seul COD ou COI, mais peut contenir un de chaque. \\ Exemple :
\begin{itemize}
    \item Je donne \textcolor{green}{un livre} (COD) \textcolor{green}{à Marie} (COI).
\end{itemize}
\subsection{Compléments optionnels}
\subsubsection{\textcolor{orange}{Compléments circonstanciels}}
Tout complément répondant à une autre question, par exemple SUR QUOI? QUAND? OÙ?, est un complément circonstanciel, i.e. un complément optionnel. \\
Exemples : 
\begin{itemize}
    \item Je mange sur la table $\rightarrow$ Je mange OÙ? = complément circonstanciel;
    \item Je pars en vacances demain $\rightarrow$ Je pars QUAND? et OÙ? = compléments circonstanciels.
\end{itemize}
Les compléments circonstanciels peuvent souvent être placés à différents endroits dans la phrase et s'accumuler. \\ Exemple : \textcolor{orange}{La semaine prochaine}, \textcolor{red}{Mario} \textcolor{blue}{louera} \textcolor{green}{une voiture} \textcolor{orange}{pour partir en vacances en Italie durant un mois}.\\
On voit bien que la phrase a toujours du sens si on enlève les parties oranges. Les compléments circonstanciels apportent simplement des informations supplémentaires et peuvent bouger. On aurait pu dire :\\
\textcolor{orange}{Pour partir en vacances en Italie durant un mois}, \textcolor{red}{Mario} \textcolor{blue}{louera}, \textcolor{orange}{la semaine prochaine}, \textcolor{green}{une voiture}.
\subsubsection{\textcolor{gray}{Négation}}
Comme en français, la négation se pose avant le verbe (ne (FR) devient non (IT)). Le "pas" français n'existe pas en italien, mais on peut toujours dire ne ... jamais (non ... mai), ne ... plus (non ... più), ne ... rien (non ... niente), etc.\\
Exemples :
\begin{itemize}
    \item \textcolor{red}{Lore et Pierre} \textcolor{gray}{ne} \textcolor{blue}{portent} \textcolor{gray}{pas} \textcolor{green}{de chaussures} -- \textcolor{red}{Lore e Pierre} \textcolor{gray}{non} \textcolor{blue}{portano} \textcolor{green}{scarpe}.
    \item \textcolor{red}{Je} \textcolor{gray}{ne} \textcolor{blue}{veux} \textcolor{gray}{plus} \textcolor{green}{de gâteau} -- \textcolor{red}{Io} \textcolor{gray}{non} \textcolor{blue}{voglio} \textcolor{gray}{più} \textcolor{green}{torta}.   
\end{itemize}
\subsubsection{\textcolor{pink}{Phrase secondaire}}
Une phrase secondaire (P2) est une phrase dans la phrase principale. Elle est introduite par une conjonction de coordination ou de subordination. La conjonction de coordination lie deux phrase distinctes, tandis que la conjonction de subordination introduit réellement une phrase placée dans une autre.\\ Exemples : 
\begin{itemize}
    \item Conjonction de coordination : \textcolor{red}{Marie} \textcolor{blue}{mange} \textcolor{green}{une pomme} \textcolor{pink}{et} \textcolor{blue}{boit} \textcolor{green}{de l'eau} -- \textcolor{red}{Maria} \textcolor{blue}{mangia} \textcolor{green}{una mela} \textcolor{pink}{e} \textcolor{blue}{beve} \textcolor{green}{dell'acqua}.
    \item Conjonction de subordination : \textcolor{red}{Pierre} \textcolor{blue}{pense} \textcolor{pink}{que} \textcolor{red}{Marie} \textcolor{blue}{mange} \textcolor{green}{une pomme} -- \textcolor{red}{Pietro} \textcolor{blue}{pensa} \textcolor{pink}{che} \textcolor{red}{Maria} \textcolor{blue}{mangia} \textcolor{green}{una mela}.
\end{itemize}
Une P2 peut être placée dans n'importe quel groupe de mot de la phrase : dans le sujet, un complément du verbe ou un complément circonstanciel. 
\subsection{Récapitulatif}
\begin{center}
    \textcolor{orange}{(CC)} + \textcolor{red}{Sujet} + \textcolor{gray}{(Négation)} + \textcolor{blue}{Verbe} + \textcolor{gray}{(Suite de la négation)} + \textcolor{orange}{(CC)} + \textcolor{green}{COD et/ou COI} + \textcolor{orange}{(CC)} + \textcolor{pink}{(Phrase secondaire)}
\end{center}
Il existe des subtilités et exceptions, par exemple avec la pronominalisation des compléments du verbe, mais la structure générale est celle-ci. 
\section{Accents}
Il existe des catégories de mots selon l'accent tonique:
\begin{itemize}
    \item tronche : accent sur la dernière syllabe (virtù, perché, caffè, città,...)\footnote{Les accents s'écrivent toujours sur la dernière syllabe.};
    \item piane : accent sur l'avant-dernière syllabe (sapòne, tènda, mangiàre, capìto,...)\footnote{Ici et pour les catégories suivantes, l'accent n'est écrit que pour illustrer, le mot réel ne le contient pas.};
    \item sdrucciole : accent sur l'antépénultième syllabe (tàvolo, mènsile, gòndola,...);
    \item bisdrucciole : accent sur la quatrième syllabe en partant de la fin (telèfonami, dìtelemo,...);
    \item trisdruciole : accent sur la cinquième syllabe en partant de la fin (rècitamelo,...).
\end{itemize}
\section{Terminaisons des noms et adjectifs}
\begin{center}
    \begin{tabular}{c|c|c}
        Singulier & Pluriel & Cas\\
        \hline
        -o & -i & Masculin \\
        -a & -e & Féminin \\
        -e & -i & Masculin ou féminin\\
    \end{tabular}
\end{center}
\begin{itemize}
    \item [$\rightarrow$] Remarque : Si un -cia ou un -gia est précédé d'une consonne, le i tombe au pluriel (-ce, -ge).
\end{itemize}
L'adjectif vient en général après le nom.
\subsection{Irréguliers}
\begin{center}
    \begin{tabular}{c|c|c}
        Singulier & Pluriel & Traduction\\ \hline 
        L'orecchio & Le orecchie & l'oreille \\
        Il braccio & Le braccia & le coeur \\
        Il ginocchio & le ginocchia & le genou\\
        Il dito & le dita & le doigt\\
        Il centinaio & le centinaia & la centaine \\
        Il migliaio & le migliaia & le millier \\
        Il paio & le paia & le couple \\
        L'uovo & le uova & l'oeuf \\
        L'uomo & gli uomini & l'homme\\
        Il bue & i buoi & le boeuf\\
        Il tempio & i templi & le temple \\
        Il dio & gli dei & le dieu\\
        L'ala & le ali &l'aile \\
        la mano & le mani & la main\\
        l'arma & le armi & l'arme\\
    \end{tabular}
\end{center}
\section{Déterminants}
\begin{center}
    \begin{tabular}{c|c|c|c|c}
        \multicolumn{2}{c|}{Déterminés} & \multicolumn{2}{c|}{Indéterminés} & \\ \hline
        Singulier & Pluriel & Singulier & Pluriel & \\
        La & Le & Una & Delle & Féminin commençant par une consonne \\ 
        L' & Le & Un' & Delle & Féminin commençant par une voyelle \\ 
        Lo & Gli & Uno & Degli & Masculin commençant par s+consonne, xyz, psi, etc\\ 
        L' & Gli & Un & Degli & Masculin commençant par une voyelle \\ 
        Il & I & Un & Dei & Masculin \\ 
    \end{tabular}    
\end{center}
\section{Prépositions}
\subsection{Prépositions utilisées avec l'article}
\begin{figure}[H]
    \centering
    \includegraphics[width = \textwidth]{img/prep.png}
\end{figure}
Les prépositions con, per, tra, fra sont utilisées avec article séparé. 
\section{Adjectifs et pronoms possessifs}
\begin{center}
    \begin{tabular}{c|c}
        Pronom & Correspondance\\ \hline
        il mio/ il tuo/ il suo/ il nostro/ il vostro/ il loro & uno \\
        la mia/ la tua/ la sua/ la nostra/ la vostra/ la loro & una\\
        i miei/ i tuoi/ i suoi/ i nostri/ i vostri/ i loro & i\\
        le mie/ le tue/ le sue/ le nostre/ le vostre/ le loro & le\\
    \end{tabular}
\end{center}
\begin{itemize}
    \item [$\rightarrow$] On ne met pas l'article s'il s'agit d'un membre de la famille au singulier, SAUF pour loro et si le nom est modifié par un adjectif diminiutif/"augmentant", e.g. la mia cara sorella, il mio fratellino.
\end{itemize}
\section{Pronoms}
\begin{center}
    \begin{tabular}{c|c|c|c|c|c}
        Pronom sujet & \multicolumn{2}{c|}{Pronom direct} & \multicolumn{2}{c|}{Pronom indirect} & Pronom réfléchi \\ \hline 
        & Faible & Fort & Faible & Fort & \\ \hline 
        io & mi & me & mi & a me & mi\\
        tu & ti & te & ti & a te & ti\\
        lui/lei/Lei & lo/la/La & lui/lei/Lei & gli/le/Le & a lui/a lei/a Lei & si/Si\\
        noi & ci & noi & ci & a noi & ci\\
        voi & vi & voi & vi & a voi & vi\\
        loro & li/le & loro & gli & a loro & si\\
    \end{tabular}
\end{center}
\begin{itemize}
    \item [$\to$] Remarque : Lorsque lo/la suit un auxiliaire, il devient l'apostrophe l' (e.g. l'ho visto). Cela n'est pas vrai avec les pronoms du pluriel.
    \item [$\to$] Remarque : la forme forte est utilisée pour insister sur le pronom.
\end{itemize}
\subsection{Usage du "ci"}
\begin{itemize}
    \item "Ci" peut être un substitut de lieu : vado in palestra (je vais à la salle de sport) $\rightarrow$ ci vado (j'y vais);
    \item "Ci" peut remplacer "a noi" comme dans le tableau ci-dessus;
    \item "Ci" est aussi le pronom réfléchi de "noi";
    \item "Ci" dans c'è et ci sono, pour "il y a";
    \item "Ci" devient "ce" quand il y a un autre pronom : "Hai il libro?" $\rightarrow$ "Ce l'ho";
    \item "Ci" dans des expressions :
    \begin{itemize}
        \item [$\bullet$] Non ci credo : j'y crois pas;
        \item [$\bullet$] Ci penso io : je m'en occupe;
        \item [$\bullet$] Ci tengo molto : j'y tiens beaucoup;
        \item [$\bullet$] Ci vuole tempo : ça prend du temps;
    \end{itemize}
\end{itemize}
\subsection{Pronoms combinés}
Les pronoms peuvent être combinés comme en français : "Je te l'ai donné" devient "Te lo ho dato". L'ordre est toujours Indirect + Direct. 
\begin{center}
\setlength{\tabcolsep}{10pt} 
\renewcommand{\arraystretch}{1.3} 
\begin{tabular}{p{2.5cm}|p{1.8cm}|p{1.8cm}|p{1.8cm}|p{1.8cm}|p{1.8cm}}
    \textbf{Indiretto con diretto} & lo & la & li & le & ne \\ \hline 
    mi $\rightarrow$ & me lo & me la & me li & me le & me ne\\ 
    ti $\rightarrow$ & te lo & te la & te li & te le & te ne\\
    gli/le $\rightarrow$ & glielo & gliela & glieli & gliele & gliene\\
    ci $\rightarrow$ & ce lo & ce la & ce li & ce le & ce ne\\
    vi $\rightarrow$ & ve lo & ve la & ve li & ve le & ve ne\\
\end{tabular}
\end{center}
\section{Anche - Neanche}
"Anche" signifie "aussi", tandis que "neanche" signifie "non plus". Ils s'utilisent comme en français et le sens de la phrase dépend d'où ils sont placés. 
\section{Connecteurs logiques}
\begin{center}
\begin{tabular}{|c|c|c|c|}
\hline
\multicolumn{2}{|c|}{Introduction} & \multicolumn{2}{c|}{Opposition} \\ \hline
Prima di tutto & Premièrement & Ciononostante & néanmoins \\ 
Per prima cosa & pour commencer & Invece & au contraire \\ 
 &  & Eppure & pourtant \\ \hline

\multicolumn{2}{|c|}{Liaison} & \multicolumn{2}{c|}{But} \\ \hline
Pertanto/percio & donc & Al fine di & afin de \\
Anche & aussi & Con l'obiettivo di & avec l'objectif de \\
Cioè & C'est-à-dire & Con l'intenzione di & avec l'intention de \\
Pero & toutefois & Dapprima & d'abord \\
Nonchè & aussi bien que &  &  \\
Nello stesso modo & de la même façon &  &  \\ \hline

\multicolumn{2}{|c|}{Approfondissement} & \multicolumn{2}{c|}{Temps} \\ \hline
Poi & ensuite & Da sempre & toujours \\
Ci sono cose da dire & il y a des choses à dire & Ormai & désormais \\
Per continuare & pour continuer & A volte & quelques fois \\
Ne deriva che & il s'ensuit que & Ora/adesso & maintenant \\
A patto che & du moment que & in questo momento & à l'heure actuelle \\
Una volta & une fois que & In questo istante & à cet instant \\ \hline

\multicolumn{2}{|c|}{Expansion explicative} & \multicolumn{2}{c|}{Lieu} \\ \hline
Tuttavia & toutefois & sotto & sous \\
A causa di & à cause de & All'interno & à l'intérieur \\
Infatti & en fait & All'externo & à l'extérieur \\
Ad esempio & par exemple & Davanti & devant \\
In altre parole & en d'autres termes & Dietro & derrière \\
tra l'altro & de plus & Qui & ici \\ \hline

\multicolumn{2}{|c|}{Similitude} & \multicolumn{2}{c|}{Cause} \\ \hline
è lo stesso di & c'est la même chose que & A causa di & à cause de \\
è simile & c'est similaire & poiché & tant que \\ 
& & Di conseguenza & en conséquence \\
& & Per finire & pour finir \\ \hline
\end{tabular}
\end{center}

\chapter{Divers}
\underline{Fichiers :}
\begin{itemize}
    \item \texttt{preposizioni riassunto}
    \item \texttt{Espressioni con Essere Fare Avere}
    \item Molto, troppo, poco s'accordent avec le nom quand ils sont adjectifs. 
\end{itemize}
\section{Heure}
\begin{center}
    \begin{tabular}{c|c}
        è mezzogiorno  & il est midi\\
        è mezzanotte & il est minuit\\
        è l'una & il est 1h\\
        sono le X & il est Xh (X>1)\\
        sono le X meno Y & il est Xh moins Y.\\
        sono le X e Y & il est XhY.\\
        sono le X e mezza/o & il est Xh30.\\
        sono le X e un quarto & il est Xh15.\\
        sono le X di mattina & il est Xh du matin.\\
        sono le X di sera & il est Xh du soir.\\
        sono le X in punto & il est Xh pile.\\
    \end{tabular}
\end{center}
\section{Vocabulaire}
Le lien Quizlet pour le vocabulaire de LITAL1100 est \href{https://quizlet.com/be/1006471421/lital1100-q2-flash-cards/}{ici}.\\
Le lien Quizlet pour le vocabulaire de LITAL1300 est \href{https://quizlet.com/be/1083840558/lital1300-flash-cards/?i=2qogk5&x=1jqt}{ici}.
\end{document}