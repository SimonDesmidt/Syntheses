\documentclass[12pt, openany]{report}
\usepackage[utf8]{inputenc}
\usepackage[T1]{fontenc}
\usepackage{amsmath,amsfonts,amssymb}
\usepackage{amssymb}
\usepackage{multicol}
\usepackage[a4paper,left=2.5cm,right=2.5cm,top=2.5cm,bottom=2.5cm]{geometry}
\usepackage[english]{babel}
\usepackage{libertine}
\usepackage{graphicx}
\usepackage{wrapfig}
\usepackage{algorithm}
\usepackage{algpseudocode}
\usepackage{float}
\usepackage{enumitem}
\usepackage{pythonhighlight}
\usepackage[]{titletoc}
\usepackage{empheq}
\usepackage{titlesec}
\usepackage{mathpazo}
\usepackage{xfrac}
\usepackage{textcomp}
\usepackage{mathtools}
\usepackage{caption}
\usepackage{tabularray}
\usepackage{subcaption}
\usepackage[bottom]{footmisc}
\usepackage{pdfpages}
\usepackage{tabularx}
\usepackage {tikz}
\usetikzlibrary{positioning}
\usepackage{amsthm}
\usepackage[skins]{tcolorbox}
\titleformat{\chapter}[display]
  {\normalfont\bfseries}{}{0pt}{\Huge}
\usepackage{hyperref}
\newcommand{\hsp}{\hspace{20pt}}
\newcommand{\HRule}{\rule{\linewidth}{0.5mm}}
\newcommand{\R}{\mathbb{R}}
\newcommand{\C}{\mathbb{C}}
\newcommand{\Oo}{\mathcal{O}}
\newcommand{\E}{\mathbb{E}}
\newcommand{\J}{\mathbf{J}}
\newcommand{\He}{\mathbf{H}}
\newcommand{\Lo}{\mathcal{L}}
\theoremstyle{definition}
\newtheorem{thm}{Theorem}[chapter]
\newtheorem{definition}[thm]{Definition}
\newtheorem{lem}[thm]{Lemma}

\hbadness=100000
\begin{document}
\begin{titlepage}
    \begin{sffamily}
    \begin{center}
        \includegraphics[scale=0.3]{img/page_de_garde.jpg} \\[1cm]
        \HRule \\[0.4cm]
        { \huge \bfseries LINMA2472 - Algorithm in data science \\[0.4cm] }
    
        \HRule \\[1.5cm]
        \textsc{\LARGE Simon Desmidt}\\[1cm]
        \vfill
        \vspace{2cm}
        {\large Academic year 2025-2026 - Q1}
        \vspace{0.4cm}
         
        \includegraphics[width=0.15\textwidth]{img/epl.png}
        
        UCLouvain\\
    
    \end{center}
    \end{sffamily}
\end{titlepage}

\setcounter{tocdepth}{1}
\tableofcontents
\chapter{Automatic differentiation}
The Automatic differentiation is an algorithmic technique to compute automatically the derivative (gradient) of a function defined in a computer program. Unlike symbolic differentiation (done by hand) and numerical  differentiation (finite difference approximation), automatic differentiation exploits the fact that every function can be decomposed into a sequence of elementary operations (addition, multiplication, sine, exponential, etc.) and so that we can apply the chain rule to compute the derivative of the whole function. Thus we can compute the gradient of a function exactly and efficiently.\\ 
Automatic differentiation is widely used in machine learning because for the neural networks, we need to compute the gradient of a loss function with respect to the parameters of the model (weights and biases) to update them during the training process and it would be difficult to compute this manually for each node.\\
\section{Chain rule}
There is two ways to apply the chain rule to compute the gradient of a function: forward differentiation and backward differentiation. Suppose that we have a composition of $m$ functions. The chain rule gives us:
\begin{equation}
  f'(x) = f_m'(f_{m-1}(f_{m-2}(...f_1(x)...))) \cdot ... \cdot f_2'(f_1(x)) \cdot f_1'(x)
\end{equation}
Let's define:
\begin{equation}
  \begin{cases}
    s_0 &= x \\
    s_k &= f_k(s_{k-1})
  \end{cases}
\end{equation}
We thus get:
\begin{equation}
  f'(x) = f'_m(s_{m-1}) \cdot ...
  \cdot f'_2(s_1) \cdot f'_1(s_0)
\end{equation}
Based on this, we can define the forward and backward differentiation algorithms.
\section{Forward differentiation}
Also called forward mode, this algorithm consists in propagating forward the derivative and the values at the same time. The fact of propagating the values forward is called a \textbf{forward pass}. It can be represented by this graph where the blue part represents the values and the green part the derivatives:
\begin{figure}[H]
    \centering
    \includegraphics[width=\textwidth]{img/forward_diff.png}
    \caption{Forward differentiation}
    \label{fig:forward_diff}
\end{figure}
And it can be computed with the following recurrence relation:
\begin{equation}
  \begin{cases}
    t_0 &= 1 \\
    t_k &= f'_k(s_{k-1}) \cdot t_{k-1}\\
  \end{cases}
\end{equation}
It is simple to implement and very efficient for functions with a small number of input variables. However, it becomes inefficient for functions with a large number of input variables because we need to compute the derivative for each input variable separately. So in practice for neural networks where we have a large number of input variables (weights and biases), we use the backward differentiation.
\section{Backward differentiation}
Also called backward mode, this algorithm consists in propagating the derivative backward and the values forward at the same time. The fact of propagating the derivative backward is called a \textbf{backward pass}. It can be represented by this graph where the blue part represents the values and the orange part the derivatives:
\begin{figure}[H]
    \centering
    \includegraphics[width=\textwidth]{img/backward_diff.png}
    \caption{Backward differentiation}
    \label{fig:backward_diff}
\end{figure}
The idea is to compute all the intermediate values $s_k$ in a forward pass and then compute the derivatives $r_k$ based on the output in a backward pass. It can be computed with the following recurrence relation:
\begin{equation}
  \begin{cases}
    r_m &= 1 \\
    r_k &= r_{k+1} \cdot f'_{k+1}(s_{k})\\
  \end{cases}
\end{equation}
This method is more complex to implement but it is very efficient for functions with a large number of input variables and a small number of output variables typically 1, the loss function.
\section{Computational graph and multivariate differentiation}
\subsection{Computational graph}  
To represent the computation of a function, we can use a computational graph. It is a directed acyclic graph where the nodes represent the operations and the edges represent the variables. For example, consider the function with $f_1(x)=x=s_1$ and $f_2(x)=x^2=s_2$:
\begin{equation}
  f_3(s_1,s_2) = s_1 + s_2 = x + x^2
\end{equation}
The computational graph is:\\
\begin{center}
  \begin{tikzpicture}[
    roundnode/.style={circle, draw=green!60, fill=green!5, very thick, minimum size=7mm},]
    %Nodes
    \node[roundnode] (x1) {$s_1=x$};
    \node[roundnode] (square) [right=of x1] {$f_2=x^2$};
    \node[roundnode] (sum) [right=of square] {$f_3=f_2+s_1$};
    
    % %Lines
    \draw[->] (x1.north) .. controls +(up:7mm) and +(up:7mm).. (sum.north);
    \draw[->] (x1.east) -- (square.west);
    \draw[->] (square.east) -- (sum.west);
  \end{tikzpicture}
\end{center}
\subsection{Multivariate differentiation}
Let's consider the function of the computational graph above:
\begin{equation}
  f_3(f_1(x),f_2(x)) = s_3 = f_1(x) + f_2(x) = s_1 + s_2 = x + x^2 
\end{equation}
following the chain rule, we have:
\begin{equation}
  f'_3(x) = \frac{\partial f_3}{\partial s_1} \frac{\partial s_1}{\partial x} + \frac{\partial f_3}{\partial s_2} \frac{\partial s_2}{\partial x}
\end{equation}
For the forward automatic differentiation, we work the same way as before, we propagate the values and the derivatives forward. But when we have a node with multiple inputs, we need to use formula derivated from the chain rule. For the function $f_3$ that we want to evaluate in $x=3$, we will have:
\begin{equation}
	\begin{cases}
		t_0 &= 1 \\
		t_1 &= f'_1(x) \vert_{x=3} \cdot t_0 = 1\\
		t_2 &= f'_2(x) \vert_{x=3} \cdot t_0 = 6\\
		t_3 &= \frac{\partial f_3}{\partial s_1} \vert_{x=3} \cdot t_1 + \frac{\partial f_3}{\partial s_2} \vert_{x=3} \cdot t_2 = 7\\
	\end{cases}
\end{equation}
For the backward automatic differentiation, first we need to initialize the gradient accumulator to 0.
\begin{equation}
	\frac{\partial s_3}{\partial s_1} = \frac{\partial s_3}{\partial s_2} = \frac{\partial s_3}{\partial x} = 0
\end{equation}
Then we compute the intermediate values in a forward pass:
\begin{equation}
	\begin{aligned}
		\frac{\partial s_3}{\partial s_1} &+= 1 \Rightarrow \frac{\partial s_3}{\partial x} += 1 \cdot 1 \vert_{x=3}\\		
		\frac{\partial s_3}{\partial s_2} &+= 1 \Rightarrow \frac{\partial s_3}{\partial x} += 1 \cdot 2x \vert_{x=3}\\
	\end{aligned}
\end{equation}
Finally we get:
\begin{equation}
	\frac{\partial s_3}{\partial x} = 7
\end{equation}
\section{Jacobian computation}
When doing the forward and backward mode, we compute the Jacobian matrix of the function. 
Using this Jacobian we can do the forward mode like this:
\begin{equation}
  J_f(x) \cdot v \qquad \qquad \text{(JVP)}
\end{equation}
where $v$ is a vector of size $n$ (number of input variables)
and the backward mode like this:
\begin{equation}
  v^T J_f(x) \qquad \qquad \text{(VJP)}
\end{equation}
Consider a function $f:\R^n \rightarrow \R^m$ then computing the full Jacobian requires $n$ forward passes (JVP) or $m$ backward passes (VJP). Therefore,
\begin{itemize}
  \item If $n \ll m$, we use the forward mode because it's faster
  \item If $n \gg m$, we use the backward mode because it's faster
  \item If $n \approx m$, we can use either mode
\end{itemize}
\section{Memory usage}
The forward mode only needs to store the current value and the current derivative, so the memory usage is relatively constant.\\
\begin{figure}[H]
    \centering
    \includegraphics[width=\textwidth]{img/fd_mem.png}
    \caption{Forward mode memory usage}
    \label{fig:fd_mem}
\end{figure}
However, the backward mode needs to store all the intermediate values to compute the derivatives in the backward pass so the memory usage will first increase then reduce when we will start to use the derivatives previously computed.\\
\begin{figure}[H]
    \centering
    \includegraphics[width=\textwidth]{img/bd_mem.png}
    \caption{Backward mode memory usage}
    \label{fig:bd_mem}
\end{figure}
So the forward mode is more memory efficient than the backward mode. However, this factor may be less significant than the number of operations performed (JVP and VJP).
\section{new part idk}

\section{Second order AD}
And what happens when we take a look at the second order derivative? Remember that we can compute the Hessian $\nabla^2 f(x)$ like this:
\begin{equation}
  (\nabla^2 f(x))_{ij} = \frac{\partial^2 f(x)}{\partial x_i \partial x_j}
\end{equation}
Or like this:
\begin{equation}
  \nabla^2 f(x) = J_{\nabla f}(x)
\end{equation}
We can easily see that with this definition, we can use an AD for the gradient and for the Jacobian to compute the Hessian. And because we can separate these two computations in distinct AD, it means that we are not forced to use the same mode for both. So we can do either forward or backward for both computations without taking care of the mode used for the other computation.\\

Based on the chain rule, let's rewrite $\frac{\partial^2 (f_2 \circ f_1)}{\partial x_i \partial x_j}$ into something more suitable for AD (consider $\partial f$ as the gradient of $f$):
\begin{equation}
  \begin{aligned}
    \frac{\partial^2 (f_2 \circ f_1)}{\partial x_i \partial x_j} &= \frac{\partial}{\partial x_j} \left( \frac{ \partial (f_2 \circ f_1)}{\partial x_i} \right) \\
    &= \frac{\partial}{\partial x_j} \left( \partial f_2 \frac{ \partial f_1}{\partial x_i} \right) \\
    &= \left( \partial^2 f_2 \frac{\partial f_1}{\partial x_j} \right) \frac{\partial f_1}{\partial x_i} + \partial f_2 \frac{\partial^2 f_1}{\partial x_i \partial x_j} 
  \end{aligned}
\end{equation}
Introducing the variables $\J_k = \partial f_k$ and $\He_{kj} = \frac{\partial}{\partial x_j} \J_k = \partial^2 f_k \frac{\partial f_{k-1}}{\partial x_j}$, and so we get:
\begin{equation}
  \frac{\partial^2 (f_2 \circ f_1)}{\partial x_i \partial x_j} = \He_{2j} \frac{\partial f_1}{\partial x_i} + \J_2 \frac{\partial^2 f_1}{\partial x_i \partial x_j}
\end{equation}
We can now define four different ways to compute the Hessian depending on the mode used for each computation.
\subsection{Forward on forward}
Define $Dual(s_1,t_1)$ with $s_1 = Dual(f_1(x), \frac{\partial f_1}{\partial x_j})$ and $t_1 = Dual(\frac{\partial f_1}{\partial x_i}, \frac{\partial^2 f_1}{\partial x_i \partial x_j})$ then we can have this algorithm:
\begin{enumerate}
  \item Compute $s_2 = f_2(s_1) = (f_2(f_1(x)), \J_2 \frac{\partial f_1}{\partial x_j})$
  \item Compute $\J_{f_2}(s_1)$ which gives $Dual(\J_2, \He_{2j})$
  \item Compute \begin{equation} \begin{aligned} t_2 &= \J_{f_2}(s_1) t_1 \\ &= Dual(\J_2, \He_{2j}) Dual(\frac{\partial f_1}{\partial x_i}, \frac{\partial^2 f_1}{\partial x_i \partial x_j}) \\ &= Dual(\J_2 \frac{\partial f_1}{\partial x_i}, \J_2 \frac{\partial^2 f_1}{\partial x_i \partial x_j} + \He_{2j} \frac{\partial f_1}{\partial x_i}) \end{aligned} \end{equation}
  \item Repeat 
\end{enumerate}
All that can be resumed as these two equations with $g_k(x) = f_k \circ \dots \circ f_1$:
\begin{equation}
  \begin{cases}
    s_k &= Dual(g_k(x), \frac{\partial g_k}{\partial x_j})\\
    t_k &= Dual(\frac{\partial g_k}{\partial x_i}, \frac{\partial^2 g_k}{\partial x_i \partial x_j})
  \end{cases}
\end{equation}
\subsection{Forward on reverse}
First the forward pass works the same as forward on forward, given $s_1 = Dual(f_1(x), \frac{\partial f_1}{\partial x_j})$ 
\begin{enumerate}
  \item Compute $s_2 = f_2(s_1) $
  \item Compute $\J_{f_2}(s_1)$ which gives $Dual(\J_2, \He_{2j})$
\end{enumerate}
Then the backward pass, given $r_2 = Dual((r_2)_1, (r_2)_2)$ compute:
\begin{equation}
  \begin{aligned}
    r_2 \J_2 &= Dual((r_2)_1, (r_2)_2) Dual(\J_2, \He_{2j})\\ &= Dual((r_2)_1 \J_2, (r_2)_1 \He_{2j} + (r_2)_2 \J_2)
  \end{aligned}
\end{equation}
Given the last equation, we get the recurrence relation:
\begin{equation}
  r_k = Dual(\frac{\partial f}{\partial s_k}, \frac{\partial^2 f}{\partial s_k \partial x_j})
\end{equation}
\subsection{Reverse on forward}
First we set up the forward pass, given $s_1 = Dual(f_1(x), \frac{\partial f_1}{\partial x_i })$ (notice the difference with the previous modes ($j \to i$)). Then we compute:
\begin{enumerate}
  \item Compute $s_2 = f_2(s_1) = Dual(f_2(s_1), \J_2 \frac{\partial f_1}{\partial x_i})$
  \item The reverse mode computes the local Jacobian \begin{equation}
    \frac{\partial s_2}{\partial s_1} = \begin{bmatrix}
      \J_2 & 0\\
      \He_{2i} & \J_2
    \end{bmatrix} % = \frac{\partial ((s_2)_1,(s_2)_2)}{\partial ((s_1)_1,(s_1)_2)}
  \end{equation} 
\end{enumerate}
Then the backward pass, which gives us:
\begin{equation}
  \begin{cases}
    (r_1)_1 = (r_2)_1 \J_2 + (r_2)_2 \He_{2i}\\
    (r_1)_2 = (r_2)_2 \J_2
  \end{cases}
\end{equation}
Which gives us the solution of recurrence equation:
\begin{equation}
  r_k = Dual(\frac{\partial f}{\partial s_k}, \frac{\partial^2 f}{\partial s_k \partial x_j})
\end{equation}
\subsection{Reverse on reverse}
First we need to set up the forward pass, so we have $s_2 = f_2(s_1)$, again we need to compute its jacobian $\J_2 = \frac{\partial s_2}{\partial s_1}$. Then we need compute the backward pass with $r_1 = r_2 \J_2$. Then we need to compute again the backward pass, in order to get the second order derivative. Let $\Dot{r}_k$ be the second order reverse tangent for $r_k$, then we have:
\begin{equation}
  \begin{aligned}
    \Dot{r}_2 &= \J_2 \Dot{r}_1  \\
    \Dot{s}_1 &= (r_2 \partial^2 f_2(s_1))\Dot{r}_1 + \Dot{s}_2 \J_2
  \end{aligned}
\end{equation}

So we can get the solution of the recurrence relation with $\Dot{s}_0 = e_i$:
\begin{equation}
  \begin{cases}
    r_k = \J_K \dots \J_{k+1}\\
    \Dot{r}_k = \J_k \dots \J_{1} e_i\\
    (r_k \partial^2 f_k) \Dot{r}_{k-1} = r_k (\partial^2 f_k \Dot{r}_{k-1})\\
    \qquad \qquad \qquad = r_k \He_{ki}\\
    \Dot{s}_k = \sum_{k=1}^{K} r_k \He_{ki} \J_{k-1} \dots \J_1
  \end{cases}
\end{equation}

\chapter{Neural networks}
% \textcolor{red}{TODO link tangent with neural networks}\\
Neural networks are a class of machine learning models inspired by the structure and function of the human brain. They are composed of layers of interconnected nodes (neurons) that process and transmit information.\\
First let's define some variables:
\begin{itemize}
  \item $X$: input data (matrix)
  \item $y$: target data
  \item $W_k$: weights matrix at layer $k$
  \item $b_k$: bias vector at layer $k$
  \item $\sigma$: activation function (ReLU, sigmoid, etc)
  \item $\ell(.)$: loss function 
  \item $H$: number of hidden layers
  \item $S_i$: intermediate state  
\end{itemize} 
To propagate and update the information through the network we use forward pass and backward pass and so automatic differentiation.\\
We can describe the forward pass of a neural network in two equivalent ways:
\begin{equation}
  \begin{aligned}
    &\text{Right to left:} \qquad &&\text{Left to right:}\\
    &S_0 = X \qquad &&S_0 = x\\
    &S_{2k-1} = W_k S_{2k-2} + b_k \qquad &&S_{2k-1} = S_{2k-2}W_k + b_k\\
    &S_{2k} = \sigma(S_{2k-1}) \qquad &&S_{2k} = \sigma(S_{2k-1})\\
    &S_{2H+1} = W_{k+1}S_{2H} \qquad &&S_{2H+1} = S_{2H}W_{k+1}\\
    &S_{2H+2} = \ell(S_{2H+1}, Y) \qquad &&S_{2H+2} = \ell(S_{2H+1}, Y)\\ 
  \end{aligned}
\end{equation}
The principal differences is that the weights acts on the left or on the right of the data, it can be useful depending whether you represents inputs as rowvectors or columnvectors.\\
It can be represented by this computational graph:\\
\begin{figure}[H]
    \centering
    \includegraphics[width=\textwidth]{img/nn_fd.png}
    \caption{Neural network forward pass}
    \label{fig:nn_fd}
\end{figure}
And the backward pass can be represented like this:\\
\begin{figure}[H]
    \centering
    \includegraphics[width=\textwidth]{img/nn_bd.png}
    \caption{Neural network backward pass}
    \label{fig:nn_bd}
\end{figure}
\section{Autoregressive models}
An Autoregressive model wants to predict the next values in a sequence based on its previous values. Given a sequence of $n_{cxt}$ (context) past vectors $x_{-1}, x_{-2}, \ldots, x_{-n_{cxt}}$, the model aims to predict the next vector $x_0$ and maybe more $x_1$. Example, we want to predict $x_0$ and $x_1$ based on the past:
\begin{equation}
  \begin{aligned}
    p(x_0, x_1 | x_{-1}, \ldots, x_{-n_{cxt}}) &= p(x_0 | x_{-1}, \ldots, x_{-n_{cxt}}) \cdot p(x_1 | x_0, x_{-1}, \ldots, x_{-n_{cxt} + 1}, x_{-n_{cxt}}) \\
    &\approx p(x_0 | x_{-1}, \ldots, x_{-n_{cxt}}) \cdot p(x_1 | x_0, x_{-1}, \ldots, x_{-n_{cxt} + 1})
  \end{aligned} 
\end{equation}
So the models aims to predict the probability distribution of the next vector $x_0$ with $\hat{p}(x_0 | X)$ with $X$ that concatenates all the context vectors. And then we use the cross-entropy to measure how well the predicted distribution $\hat{p}$ matches the true distribution $p$:
\begin{equation}
  \Lo(X) = - \sum_{x_0} p(x_0 | X) \log(\hat{p}(x_0 | X))
\end{equation}
\section{Tokenization}
How can we use this autoregressive model for LLM (Large Language Models) for example ? Consider that we have $n_{cxt}$ characters of context and we want to predict the next characters, this means that we would have a one-hot encoding in $\R^{26}$, which means $n_{voc} = 26$. This means that $n_{cxt}$ should be a very large number, but this is annoying because transormers have a quadratic complexity in $n_{cxt}$.\\ 
Another idea could be to turn each word into a one-hot encoding, but the vocabulary size is often very large (+200k words), with this we could have a relative low $n_{ctx}$, but $n_{voc}$ would be to big.\\
An intermediate solution is to use byte pair encoding algorithm which greedily merges the most frequent pairs of characters into new tokens.\\ 
Example: consider the word "abracadabra", we see that we have two frequent pairs "ab" and "ra", so we can merge them into new tokens $X$ and $Y$ respectively:
\begin{equation}
  \text{abracadabra} \rightarrow \text{XYcadXY}
\end{equation}
then we can repeat the process until we reach the desired vocabulary size.\\ The next iteration could give:
\begin{equation}
  \text{XYcadXY} \rightarrow \text{ZcadZ}
\end{equation}
This tokenization method allows us to have good trade-off between the vocabulary size $n_{voc}$ and the context size $n_{cxt}$. But if the model isn't trained enough we could have an issue which is that the model doesn't output fully constructed words, for example it could output half of a word because that what makes more sense even though the word is half complete.\\
\section{Embedding}
Consider a vocabulary of size $n_{voc}$, a bigram model and a network with $d$ layers. The model would be:
\begin{equation}
  \hat{p}(x_0 | x_{-1}) = \text{softmax}(W_d \tanh(... \tanh(W_1 x_{-1}) ... ))
\end{equation}
The matrix $W_1$ has $n_{voc}$ columns and $W_d$ has $n_{voc}$ rows, so when $n_{voc}$ is large, the model becomes very big.\\
So an idea would be to use an encoder and a decoder to reduce the sizes, it is called an embedding size $d_{emb} \ll n_{voc}$. The encoder ($C \in \R^{d_{emb} \times n_{voc}}$) maps the one-hot encoding of size $n_{voc}$ to a dense vector of size $d_{emb}$ and the decoder ($D \in \R^{n_{voc} \times d_{emb}}$) maps back the dense vector to a vector of size $n_{voc}$. So the model becomes:
\begin{equation}
  \hat{p}(x_0 | x_{-1}) = \text{softmax}(D W_d \tanh(... \tanh(W_1 C x_{-1})... ))
\end{equation}
If we choose wisely $d_{emb}$, which means much smaller than $n_{voc}$, then it is faster to compute $W_1(C x_{-1})$ than in the previous model. Moreover we are forcing $W_1C$ to be low-rank which can help to reduce overfitting but reduce the expressivness (capacity to capture a range of possible relation between input and output) of the model. It is useful to share the embedding between different input and output when $n_{ctx} > 1$, we also found that forcing $D = C^T$ appears to work well in practice.\\
\subsection{Shared embedding}
Let's investigate the case where we have $n_{cxt} > 1$. When $n_{cxt} > 1$, the encoder C is shared by all tokens, so we get this model:
\begin{equation}
  \hat{p}(x_0 | x_{-1}, \ldots, x_{-n_{cxt}}) = \text{softmax}(D W_d \tanh(... \tanh(W_1 \begin{bmatrix} C x_{-1} \\ \vdots \\ C x_{-n_{cxt}} \end{bmatrix}) ... ))
\end{equation}
We can note that now $W_1$ has $n_{cxt}d_{emb}$ columns. Assuming $d_{emb} \ll n_{voc}$ and $n_{cxt} \gg 1$, this is much smaller than the $n_{ctx}n_{voc}$ that we had before the embedding. The number of rows of $W_2$ is not affected by the embedding so it remains $n_{voc}$.\\
We could represent this model like this:\\
\begin{figure}[H]
    \centering
    \includegraphics[width=0.7\textwidth]{img/shared_embedding.png}
    \caption{Neural network with shared embedding}
    \label{fig:shared_embedding}
\end{figure}
\section{Recurrent neural networks (RNN)}
A RNN is a type of neural network that is designed to process sequential data by taking advantages of the memory of previous inputs. The idea is to have a hidden state $h_t$ that depends on the current input $x_t$ and the previous hidden state $h_{t-1}$. With this idea, the network reuses the same weights at every time step and thus gives meaning to the sequence. The equations of a simple RNN are:
\begin{equation}
  \begin{cases}
    h_{t+1} = \tanh(W h_{t} + U x_{t-1} + b) \\
    \hat{y}_t = \text{softmax}(V h_t + c)
  \end{cases}
\end{equation}
Where $W, U, V$ are weight matrices and $b, c$ are bias vectors.\\
This model presents some limitations, for example it is difficult to learn long-term dependencies because of the vanishing/exploding gradient problem. The fact that it processes a sequence step by step makes it difficult to parallelize the computations. The fact that it need to store the hidden state at each time step can be also memory-intensive for long sequences.\\
\section{Attention head}
First we need to define a numerical dictionary. Consider keys $k_i \in \R^{d_k}$, values $v_i \in \R^{d_v}$. Given a query $q \in \R^{d_k}$, we want to retrieve the value $v_i$ corresponding to the key $k_i$ that is the most similar to the query $q$. To do this, we can use the dot-product. With that definition, we can define the attention head:
\begin{equation}
  \text{Attention}(Q,K,V) = \sum_{i=1}^{n_{ctx}} \alpha_i v_i
\end{equation}
where $\alpha = \text{softmax}(<q,k_1>, \cdots, <q,k_{n_{ctx}}>)$ is the attention weight for key $k_i$. But in practice, we have vectors of queries, keys and values, each contained in matrices $Q, K, V$. So we can rewrite the attention head as:
\begin{equation}
  \text{Attention}(Q,K,V) = V\text{softmax}\left(\frac{K^TQ}{\sqrt{d_k}}\right)
\end{equation}
Where the division by $\sqrt{d_k}$ is used to prevent the dot-product from growing too large.\\
For a transformers, we could explain $Q, K, V$ like this, for each token we have a vector $q, k, v$ that represent:
\begin{itemize}
  \item $q$: What am I looking for?
  \item $k$: What do I contain?
  \item $v$: What information do I pass if I’m selected?
\end{itemize}
\subsection{Masked attention}
To ensure that the model only attends to previous tokens and not future tokens, we can use a masked attention mechanism. This is done by applying a mask to the attention weights before the softmax operation. The mask is typically a lower triangular matrix that sets the weights corresponding to future tokens to negative infinity, effectively preventing them from contributing to the attention output.\\
So with masked attention, the equation becomes:
\begin{equation}
  \begin{cases}
    M = \begin{bmatrix} 0 & 0 &  \cdots & 0 \\ -\infty & 0 &  \ddots & 0 \\ \vdots & \ddots & \ddots & \vdots \\ -\infty & -\infty &  \cdots & 0 \\ \end{bmatrix}\\
    \text{Attention}(Q,K,V) = V\text{softmax}\left(M+\frac{K^TQ}{\sqrt{d_k}}\right)
  \end{cases}
\end{equation}
\subsection{Multi-head attention}
Instead of having a single attention head, we can have multiple attention heads that can understand different aspects of the input data. Each head has its own set of weights and computes its own attention output, all in parallel. After computing all the heads, their outputs are concatenated and linearly transformed to be in the right dimension with $W^O$, this is thus gathering all of the meanings of the attention different heads to a single representation. So the multi-head attention can be defined as:
\begin{equation}
  \begin{cases}
    \text{head}_j = \text{Attention}(W_j^QQ, W_j^KK, W_j^VV) \\
    \text{MultiHead}(Q,K,V) = W^O \begin{bmatrix} \text{head}_1 \\ \text{head}_2 \\ \vdots \\ \text{head}_h \end{bmatrix}
  \end{cases}
\end{equation}
\section{Decoder-only transformer}
A decoder-only transformer is a type of transformer architecture that is used for autoregressive model. It consists of reading the tokens and predicting the next token in the sequence. In this type of transformer, the matrices $Q, K, V$ are the same, and we define them as the input tokens $CX$, $C$ being the embedding matrix. So the decoder-only transformer can be computed with $\text{MultiHead}(CX,CX,CX)$. Let's seek the different mechanisms used in a decoder-only transformer.\\
\subsection{Positional encoding}
The positional encoding is essential in a transformer model because it provides information about the order of the tokens in the input sequence. Since transformers do not have a built-in notion of sequence order like RNNs, positional encodings are added to the input embeddings to give the model a sense of position.\\
We cannot use one-hot encoding for the position because the dimension would be $n_{ctx}$ and not $d_{emb}$ as expected. The classic approach is to use sines and cosines with an angle that depends on the position of the token and the embedding dimension. The positional encoding for a token at position $pos$ and with the embedding dimension index $i$ and embedding dimension $d_{emb}$ could be defined as:
\begin{equation}
  \begin{cases}
    angle = \frac{pos}{10000^{2i/d_{emb}}} \\
    PE[pos, 2i] = \sin\left(angle\right) \\
    PE[pos, 2i+1] = \cos\left(angle\right)
  \end{cases}
\end{equation}
And then we add this positional encoding to the input embeddings before doing the attention mechanism:
\begin{equation}
  \text{MultiHead}(CX + PE, CX + PE, CX + PE)
\end{equation}
\subsection{Residual connection}
The principle of residual connection is to add the input of a layer to its output before applying the layer normalization (next subsection). This helps to mitigate the vanishing gradient problem and help the network to learn because its learning is more controlled, because we remind it of the original input.\\
\subsection{Layer normalization}
The norm of the gradient increases exponentially with the number of layers, so to prevent this we can use layer normalization. This helps stabilize the training process and improve convergence, by rescaling the data before the activation function, which means that we control the mean and the variance of the data. There is two way of doing this normalization, the batch normalization and the layer normalization. The batch normalization is difficult and not very efficient to use in transformers, so we use layer normalization. It is done like this:
\begin{equation}
  \text{LayerNorm}(y_i) = \gamma \frac{y_i - \mu_i}{\sigma_i} + \beta 
\end{equation}
With $\gamma$ and $\beta$ two learnable parameters, $\mu_i$ the mean of the elements of $y_i$ and $\sigma_i$ the standard deviation of the elements of $y_i$. Then we can add this in the transformer architecture like this:
\begin{equation}
  \text{LayerNorm}(\text{MultiHead}(CX + PE, CX + PE, CX + PE) + (CX + PE))
\end{equation}
\subsection{Feed-forward network (FFN)}
The feed-forward network consists in doing this operation:
\begin{equation}
  \text{FFN}(x) = W_2 \text{ReLU}(W_1 x + b_1) + b_2
\end{equation}
With $W_1 \in \R^{d_{ff} \times d_{emb}}$ and $W_2 \in \R^{d_{emb} \times d_{ff}}$, $d_{ff}$ being the dimension of the feed-forward network, generally $d_{ff} = 4d_{emb}$. This feed-forward network is applied to each token independently and identically. The feed-forward network is used to introduce non-linearity and increase the model's capacity to learn complex patterns in the data. It is used the memorize more complex information in the weights, which helps the model to better understand the complex relationships between tokens in the sequence.
\subsection{Transformers variation}
\textcolor{red}{To investigate ?}
\section{Performances of transformers}
Before analysing the complexity of a transformer, let us reminds the main dimension of the variables:
\begin{itemize}
  \item $n_{ctx}$: context size (number of tokens in the input sequence)
  \item $n_{voc}$: vocabulary size (number of unique tokens)
  \item $d_{emb}$: embedding size (dimension of embeddings)
  \item $d_k, d_v$: key and value dimension (per head)
  \item $d_{ff}$: feed-forward network dimension
  \item $N$: number of layers
\end{itemize}
We can thus analyse the time complexity of a transformer step by step (ignoring the mebeding step):
\begin{enumerate}
  \item Linear projections for $Q, K, V$ (with $W_j^Q, W_j^K, W_j^V$): $\Oo\left(d_k d_{emb} n_{ctx}\right)$
  \item Attention score computation ($K^TQ$): $\Oo\left((d_k+d_v)n_{ctx}^2\right)$
  \item Output projection ($W^O$): $\Oo\left(d_v d_{emb} n_{ctx}\right)$
  \item FFN: $\Oo\left(d_{emb} d_{ff} n_{ctx}\right)$
\end{enumerate}
This brings the total cost for a single layer to:
\begin{equation}
  \Oo\left( (d_k + d_v) n_{ctx}^2 + (d_k + d_v + d_{ff}) d_{emb} n_{ctx} \right)
\end{equation}
To get the total cost for $N$ layers, we just need to multiply by $N$.\\
\begin{equation}
  \Oo\left( N (d_k + d_v) n_{ctx}^2 + N (d_k + d_v + d_{ff}) d_{emb} n_{ctx} \right)
\end{equation}
Knowing that generally $d_k \approx d_v \approx d_{emb} \approx d_{ff}$, we can simplify the complexity to:
\begin{equation}
  \Oo\left( N d_{emb} n_{ctx}^2 + N d_{emb}^2 n_{ctx} \right) = \Oo\left( N d_{emb} n_{ctx} (n_{ctx} + d_{emb}) \right)
\end{equation}
Here we can see the two dominant terms of the complexity:
\begin{itemize}
  \item $N d_{emb} n_{ctx}^2$: comes from the attention score computation, it is the most critical term when $n_{ctx}$ is large.
  \item $N d_{emb}^2 n_{ctx}$: comes from the FFN and the projection, it is the most critical term when $d_{emb}$ and thus for wide models.
\end{itemize}
We can also observe that the dimension of the vocabulary $n_{voc}$ does not appear in the complexity. It is because it is only involved in the embedding complexity.
\section{Encoder-decoder transformers}







\chapter{Kernels}
\begin{figure}[H]
	\centering
	\includegraphics[width=0.5\textwidth]{img/kernels.png}
	\caption{Illustration of kernels}
\end{figure}
A kernel is a function that transforms a dataset into another, typically of higher dimension. This helps separate the nonlinear feature, to use the usual linear tools. For example, the canonical kernel is 
\begin{equation}
  r(x,y) = (x^Ty)^2
\end{equation} 
where the kernel function then is 
\begin{equation}\label{eq:example}
  \phi(x) = \begin{pmatrix}
    x_1^2 \\ x_2^2 \\ \sqrt{2}x_1x_2
  \end{pmatrix}
\end{equation}
\section{Reminders on scalar product}
\begin{itemize}
  \item [$\to$] Reminder: a Euclidean space is a finite-dimensional vector space endowed with a scalar product. 
\end{itemize}
A scalar product $\langle \cdot, \cdot \rangle_V$ verifies 
\begin{itemize}
  \item Symmetry: $\forall x,y\in V, \: \langle x,y\rangle_V = \langle y,x \rangle_V$;
  \item Definite positive: if $x\in V$ and $x\neq 0$, then $\langle x,x\rangle >0$, and if $x=0$, then $\langle x,x\rangle=0$;
  \item Bilinearity: $\forall x,y,z\in V, \: \forall \alpha,\beta \in \mathbb{F}, \langle(\alpha x+\beta y),z\rangle_V = \alpha \langle x,z\rangle_V + \beta \langle y,z\rangle_V$.
\end{itemize}
\subsection{Equivalence}
Consider a positive definite symmetric matrix $M\succ 0\in \R^{n\times n}$ with the scalar product $\langle x,y\rangle_M = x^TMy$. Any such matrix can be factored as $M=L^TL$, where $L$ is invertible. We can do a change of coordinates to re-express under the cartesian scalar product:
\begin{equation}
  \begin{cases}
    w = Lx \\
    z = Ly
  \end{cases} \Longrightarrow \langle x,y\rangle_M = \langle w,z\rangle_{\R^n}
\end{equation}
Then, all $n$-dimensional Euclidean spaces are equivalent up to a change of coordinates. 
\subsection{Hilbert space}
For functions integrable on an interval $[a,b]$, we define the scalar product as 
\begin{equation}
  \langle f,g\rangle=\int_a^b f(t)g(t)dt 
\end{equation}
\section{Kernel methods for finite sets}
\subsection{Example -- word embedding}
Text embedding has as objective to represent a text with a vector. The general idea is one-hot encoding, i.e. for a data set $X$ of $N$ words, we create one-hot vectors of dimension $N$, using the feature map (or embedding) $\phi:X\to H$. A sentence is simply the sum of those vectors, weighted by the number of occurences. From this, we can create the kernel matrix $K$, where each element $K_{ij}$ is the comparison between the basis vectors $e_i, e_j$. We suppose that the $\phi$ function has a unitary norm.
\begin{equation}
  K_{ij} = \langle \phi(x_i),\phi(x_j)\rangle_K \in [-1,1]
\end{equation}
A value of $1$ means that the words $x_i$ and $x_j$ are identical, but if the value is $-1$, their meaning are opposite. In the case of a $0$, there is no connection between the words. Now, we can calculate the Cholesky decomposition $K=L^TL$ for a new basis, and use the usual scalar product. 
\begin{itemize}
  \item [$\to$] Note: if the matrix $K$ is not invertible, we just hit a degenerate case where the $\phi(x_i)$ are not linearly independent. This is not a problem. 
\end{itemize}
\subsection{Kernel trick}
The kernel trick is to never explicitly define the function $\phi$: we can work with the matrix $K$ only, as any vector can be expressed as a linear combination of $\phi(x_i)$ and the main ingredient of linear methods is generally the scalar product, here defined only using $K$. 
\begin{equation}
  \langle \phi(y),\phi(z)\rangle = \langle \sum_{i=1}^N \alpha_i \phi(x_i), \sum_{i=1}^N \beta_i \phi(x_i)\rangle_K = \sum_{i=1}^N \sum_{j=1}^N \alpha_i \beta_j \langle \phi(x_i),\phi(x_j)\rangle_K = \alpha^T K \beta 
\end{equation}
\subsection{Reconstruction}
Given the matrix $K$, we can reconstruction the embedding ($\phi$ and $\langle \cdot,\cdot\rangle_K)$ up to a rotation. For $X$ the set of words and $N$ the dimension of the space, we have $\phi:X\to \R^N$ and we define the scalar product:
\begin{equation}
  \langle x,y\rangle_M = \langle x,y\rangle_{K^-1}
\end{equation}
Then, remembering $\phi(x_i) = K e_i$,
\begin{equation}
  \langle \phi(x_i), \phi(x_j)\rangle_M = (Ke_i)^T K^{-1}(Ke_j) = e_i^T K^TK^{-1}Ke_j = e_i^T K e_j
\end{equation}
Let us prove the "up to a rotation" part. Define 
\begin{equation}
  \begin{cases}
    \phi':X\to \R^N \\ \phi'(x) = Q\phi(x) \\ \langle x,y\rangle_{M'} = \langle x,y\rangle_{QK^{-1}Q^T}
  \end{cases}
\end{equation}
where $Q$ is a rotation matrix, i.e. $Q^TQ = I$. Then,
\begin{equation}
  \begin{aligned}
    \langle \phi'(x),\phi'(y)\rangle_{M'}&= (Q\phi(x))^T Q K^{-1} Q^T (Q\phi(y))\\
    &= \phi(x)^T K^{-1}\phi(y) = \langle \phi(x),\phi(y)\rangle_{K^{-1}}
  \end{aligned}
\end{equation}
The result is true for any rotation matrix $Q$. 
\begin{itemize}
  \item [$\to$] Note: the complexity of computing $K$ is $\mathcal{O}(N^2)$, while the Cholesky decomposition to find the new basis has a complexity of $\mathcal{O}(N^3)$. This decomposition is to be avoided if not really necessary. 
\end{itemize}
\subsection{Advantages}
Let us consider proteins. The alphabet is of size $20$ (number of existing amino-acids), and a protein is made of 4 of these. This means that our space is initially $H = \R^{20^4} = \R^{160.000}$. Consider $N=100$ proteins to classify. \\
The embedding consists in storing $N$ vectors of size $h=160.000$, meaning 1.6M scalars. However, we only need the $K$ matrix of size $N^2=10.000$, and the entries are easy to compute. This means that, using the kernel trick, i.e. working with $K$ directly, we are much more efficient when $N<<h$. 
\section{Kernels methods for continuous sets}
\subsection{Reproducibility Kernel Hilbert Space}
Consider richer sets than finite $X$, e.g. infinite or uncountable sets, with distances defined but not scalar product. Then, let $H$ be a vector space of conitnuous functions $X\to \R$. $H$ is a RKHS if it is a Hilbert space and 
\begin{equation}
  \forall x\in X, \: \exists v\in H \text{ such that } \forall f\in H\:: \: \langle v,f\rangle_H = f(x)
\end{equation}
Mathematically, the kernel function is the equivalent of the kernel matrix, but for infinite spaces:
\begin{equation}
  \phi:X\to \R^N \text{ such that } k(x,y) = \langle \phi(x),\phi(y)\rangle_K = \alpha_x^T K \alpha_y
\end{equation}
For a RKHS $H$, we verify 
\begin{equation}
  \begin{cases}
    \forall x\in X\: : \: k(x,\cdot)\in H\\
    \forall x\in X\: ,\forall f\in H, \: f(x) = \langle f,k(x,\cdot)\rangle_K
  \end{cases}
\end{equation}
The reproducing kernel property is 
\begin{equation}
  k(x,y) = \langle k(x, \cdot), k(y,\cdot)\rangle_H
\end{equation}
\subsection{Properties}
\begin{thm}
  A continuous map $k:X\times X \to \R$ is the kernel of some RKHS $H\subset \mathcal{C}(X,\R)$
  \begin{itemize}
    \item iff $k(x,y) = \langle \phi(x),\phi(y)\rangle_H$ for some feature map $\phi:X\to H$;
    \item iff, for all finite subsets $X_0\subset X$, the kernel matrix is symmetric positive definite.
  \end{itemize}
\end{thm}
\section{Polynomial kernels}
Polynomial kernels are a particular type of continuous kernels, where the feature map is a function 
\begin{equation}
  \phi:R^n \to \R^\ell 
\end{equation}
Let us take our initial example \ref{eq:example}, and generalize it to 
\begin{equation}
  k(x,y) = (x^Ty)^d  \qquad X = \R^n 
\end{equation}
Then, the dimension of the out space $H = \R^\ell$ is $\ell = \begin{pmatrix}
  n+d-1 \\d
\end{pmatrix}$. 
\end{document}