\documentclass[11pt]{article}
\pagestyle{plain}
\usepackage[utf8]{inputenc}
\usepackage[T1]{fontenc}
\usepackage{amsmath,amsfonts,amssymb}
\usepackage{amssymb}
\usepackage{multicol}
\usepackage[a4paper,left=2.5cm,right=2.5cm,top=1cm,bottom=2.5cm]{geometry}
\usepackage[english]{babel}
\usepackage{libertine}
\usepackage{gensymb}
\usepackage{tabularray}
\usepackage{graphicx}
\usepackage{steinmetz}
\usepackage{wrapfig}
\usepackage{float}
\usepackage{enumitem}
\usepackage[]{titletoc}
\usepackage{nicematrix}
\usepackage{titlesec}
\usepackage{mathtools}
\usepackage{caption}
\usepackage{subcaption}
\usepackage[bottom]{footmisc}
\usepackage{pdfpages}
\usepackage{tabularx}
\usepackage{verbatim}
\titleformat{\chapter}[display]
  {\normalfont\bfseries}{}{0pt}{\Huge}
\usepackage{hyperref}
\newcommand{\hsp}{\hspace{20pt}}
\newcommand{\HRule}{\rule{\linewidth}{0.5mm}}
\newcommand\independent{\protect\mathpalette{\protect\independenT}{\perp}}
\def\independenT#1#2{\mathrel{\rlap{$#1#2$}\mkern2mu{#1#2}}}

\usepackage{apptools}
    \titleformat{\chapter}[hang]{\bfseries\huge}{\IfAppendix{\appendixname~}{\relax}\thechapter\IfAppendix{.}{.}}{\IfAppendix{0.333em}{2pc}}{}
%\AtBeginEnvironment{}{\appendixtrue}

% this alters "before" spacing (the second length argument) to 0
\titlespacing {\chapter}{0pt}{0pt}{40pt}

\usepackage[nottoc]{tocbibind}
\usepackage{graphicx}
\usepackage{float}

\usepackage{keyval}
\usepackage{kvoptions}
\usepackage{fancyvrb}
\usepackage{ifthen}
\usepackage{calc}
\usepackage{pdftexcmds}
\usepackage{etoolbox}
\usepackage{xstring}
\usepackage{xcolor}
\usepackage{lineno}
\usepackage{tikz}
\usepackage{circuitikz}
\usetikzlibrary{patterns,arrows,decorations.pathreplacing,babel}

\usepackage[]{minted}
\newminted{python}{
    linenos=true,
    bgcolor=lightgray,
    tabsize=4,
    gobble=8,
    fontfamily=courier,
    fontsize=\small,
    xleftmargin=5pt,
    xrightmargin=5pt
}
    \title{LINMA1731 - Assignment 4}
    \author{Desmidt Simon - NOMA 19012100}
    \date{\today}
\begin{document}
\maketitle
Let $X(t)$ be a continuous WSS stochastic process. The covariance function of $X(t)$ is given by $C_X(\tau) = e^{-\tau^2/2}$ and its mean by $m_X = 1$. $X(t)$ is filtered by a LTI system of impulse response $h(t) = e^{-3t}u(t)$ to produce a new process $Y(t)$.
\begin{enumerate}
    \item Derive the expression of the power spectrum density of $X(t)$. 
\end{enumerate}
The power spectrum density of \(X(t)\) is the Fourier transform of \(X(t)\) and is defined as follows:
\begin{equation}
    \gamma_X(\omega)\int_{-\infty}^{\infty}C_X(\tau)e^{-j\omega \tau}d\tau
\end{equation}
Therefore, its expression is 
\begin{equation}
    \gamma_X(\omega) = \int_{-\infty}^{\infty} e^{-j\omega \tau} e^{-\tau^2/2}d\tau
\end{equation}
\begin{equation}
    \color{red}\boxed{\color{black}\gamma_X(\omega) = \sqrt{2\pi}e^{-\omega^2/2}}\color{black}
\end{equation}
\begin{enumerate}
    \setcounter{enumi}{1}
    \item Derive the mean of $Y(t)$. 
\end{enumerate}
The expression of the mean of \(Y(t)\), based on the mean of \(X(t)\), is 
\begin{equation}
    m_Y = m_X \int_t h(t)dt = m_XH(0)
\end{equation}
with \(h(t)\) the impulse response and \(H(\omega)\) the transfer function of the system.
\begin{equation}
    H(\omega) = \mathcal{L}(h(t)) = \frac{1}{j\omega+3} \longrightarrow H(0) = \frac{1}{3}
\end{equation}
\begin{equation}
    \color{red}\boxed{\color{black}m_Y = m_XH(0) = \frac{1}{3}}\color{black}
\end{equation}
\begin{enumerate}
    \setcounter{enumi}{2}
    \item Derive the expression of the power spectrum density of $Y(t)$.
\end{enumerate}
We can derive the expression of the psd of \(Y(t)\) based on the one for \(X(t)\):
\begin{equation}
    \gamma_Y(\omega) = |H(\omega)|^2\gamma_X(\omega) = \frac{1}{(j\omega+3)(3-j\omega)}e^{-\omega^2/2}
\end{equation}
\begin{equation}
    \color{red}\boxed{\color{black}\gamma_Y(\omega) = \frac{1}{9+\omega^2}e^{-\omega^2/2}}\color{black}
\end{equation}
\end{document}