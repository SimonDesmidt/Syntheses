\documentclass[11pt]{article}
\pagestyle{plain}
\usepackage[utf8]{inputenc}
\usepackage[T1]{fontenc}
\usepackage{amsmath,amsfonts,amssymb}
\usepackage{amssymb}
\usepackage{multicol}
\usepackage[a4paper,left=2.5cm,right=2.5cm,top=1cm,bottom=2.5cm]{geometry}
\usepackage[english]{babel}
\usepackage{libertine}
\usepackage{gensymb}
\usepackage{tabularray}
\usepackage{graphicx}
\usepackage{steinmetz}
\usepackage{wrapfig}
\usepackage{float}
\usepackage{enumitem}
\usepackage[]{titletoc}
\usepackage{nicematrix}
\usepackage{titlesec}
\usepackage{mathtools}
\usepackage{caption}
\usepackage{subcaption}
\usepackage[bottom]{footmisc}
\usepackage{pdfpages}
\usepackage{tabularx}
\usepackage{verbatim}
\titleformat{\chapter}[display]
  {\normalfont\bfseries}{}{0pt}{\Huge}
\usepackage{hyperref}
\newcommand{\hsp}{\hspace{20pt}}
\newcommand{\HRule}{\rule{\linewidth}{0.5mm}}
\newcommand\independent{\protect\mathpalette{\protect\independenT}{\perp}}
\def\independenT#1#2{\mathrel{\rlap{$#1#2$}\mkern2mu{#1#2}}}

\usepackage{apptools}
    \titleformat{\chapter}[hang]{\bfseries\huge}{\IfAppendix{\appendixname~}{\relax}\thechapter\IfAppendix{.}{.}}{\IfAppendix{0.333em}{2pc}}{}
%\AtBeginEnvironment{}{\appendixtrue}

% this alters "before" spacing (the second length argument) to 0
\titlespacing {\chapter}{0pt}{0pt}{40pt}

\usepackage[nottoc]{tocbibind}
\usepackage{graphicx}
\usepackage{float}

\usepackage{keyval}
\usepackage{kvoptions}
\usepackage{fancyvrb}
\usepackage{ifthen}
\usepackage{calc}
\usepackage{pdftexcmds}
\usepackage{etoolbox}
\usepackage{xstring}
\usepackage{xcolor}
\usepackage{lineno}
\usepackage{tikz}
\usepackage{circuitikz}
\usetikzlibrary{patterns,arrows,decorations.pathreplacing,babel}

\usepackage[]{minted}
\newminted{python}{
    linenos=true,
    bgcolor=lightgray,
    tabsize=4,
    gobble=8,
    fontfamily=courier,
    fontsize=\small,
    xleftmargin=5pt,
    xrightmargin=5pt
}
    \title{LINMA1731 - Assignment 2}
    \author{Desmidt Simon - NOMA 19012100}
    \date{\today}
\begin{document}
\maketitle

Consider the following discrete-time system with signal \(s\) and observations \(x\):
\begin{align*}
    s[k]&= \begin{pmatrix}
        3&1\\
        0&0.5\\
    \end{pmatrix} s[k-1] \qquad k\ge 0\\
    x[k]& = \begin{pmatrix}
        0.5 & 0.5
    \end{pmatrix} s[k] + w[k]
\end{align*}
where \(s[k]\in \mathbb{R}^2,x[k] \in \mathbb{R}, w[k]\sim \mathcal{N}(0,0.5)\) and \(s[-1]\sim \mathcal{N}\left(\begin{pmatrix}
    0\\
    0\\
\end{pmatrix}, \begin{pmatrix}
    2 & 0\\
    0 & 2\\
\end{pmatrix}\right)\).
\begin{enumerate}
    \item Obtain the Minimum predictor \(M(0|-1)\).
\end{enumerate}
The general formula from the vector Kalman filter equations is 
\begin{equation}
    M(n|n-1) = AM(n-1|n-1)A^T+BQB^T
\end{equation}
Here, we have 
\begin{equation}
    A = \begin{pmatrix}
        3 & 1\\
        0 & 1/2\\
    \end{pmatrix}\qquad M(-1|-1) = C_s = \begin{pmatrix}
        2 & 0\\
        0 & 2\\
    \end{pmatrix} \qquad 
    B = 0
\end{equation}
Therefore, we have 
\begin{equation}
    \color{red}\boxed{\color{black}M(0|-1) = AC_sA^T = 2AA^T = \begin{pmatrix}
        20 & 1\\
        1 & 1/2\\
    \end{pmatrix}}\color{black}
\end{equation}
\begin{enumerate}
    \setcounter{enumi}{1}
    \item Compute the Kalman gain matrix at time step \(k=0\).
\end{enumerate}
From the Kalman filter equations, we have 
\begin{equation}
    K(n) = M(n|n-1) H^T(n)\left(C(n)+H(n)M(n|n-1)H^T(n)\right)^{-1}
\end{equation}
And we know the following:
\begin{equation}
    H(n) = H = \begin{pmatrix}
        1/2 & 1/2\\
    \end{pmatrix}
    \qquad C(n) = C = 1/2\qquad M(0|-1) = \begin{pmatrix}
        20 & 1\\
        1 & 1/2\\
    \end{pmatrix}
\end{equation}
We can finally find the value of the gain matrix:
\begin{equation}
    \color{red}\boxed{\color{black}
    K(0) = M(0|-1)H^T\left(C(0)+HM(0|-1)H^T\right)^{-1} = \frac{8}{49}\begin{pmatrix}
        21/2\\
        3/4\\
    \end{pmatrix} = \begin{pmatrix}
        12/7\\
        6/49\\
    \end{pmatrix}}\color{black}
\end{equation}
\begin{enumerate}
    \setcounter{enumi}{2}
    \item Obtain a recursive relation between \(\Hat{s}(0|0)\) and \(\Hat{s}(0|-1)\).
\end{enumerate}
Still using the general equations for a vector Kalman filter, we have 
\begin{equation}
    \Hat{s}[n|n] = \Hat{s}[n|n-1] + K[n](x[n]-H[n]\Hat{s}[n|n-1])
\end{equation}
From what we have derived earlier and by using the definition of the problem, we find 

\begin{equation}
    \color{red}\boxed{\color{black}\Hat{s}(0|0) = (I-K(0)H)\Hat{s}(0|-1) + K(0)\left(HAs(-1)+w(0)\right)}\color{black}
\end{equation}
\begin{equation}
    \Hat{s}(0|0) = \begin{pmatrix}
        1/7 & -6/7\\
        -3/49 & 46/49\\
    \end{pmatrix} \Hat{s}(0|-1) + \begin{pmatrix}
        18/7 & 9/7\\
        9/49 & 9/98\\
    \end{pmatrix} s(-1) + \begin{pmatrix}
        12/7\\
        6/49\\
    \end{pmatrix} w(0)
\end{equation}
\end{document}