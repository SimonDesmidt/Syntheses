\documentclass[12pt, openany]{report}
\usepackage[utf8]{inputenc}
\usepackage[T1]{fontenc}
\usepackage{amsmath,amsfonts,amssymb}
\usepackage{amssymb}
\usepackage{multicol}
\usepackage[a4paper,left=2.5cm,right=2.5cm,top=2.5cm,bottom=2.5cm]{geometry}
\usepackage[english]{babel}
\usepackage{libertine}
\usepackage{graphicx}
\usepackage{wrapfig}
\usepackage{algorithm}
\usepackage{algpseudocode}
\usepackage{float}
\usepackage{enumitem}
\usepackage{pythonhighlight}
\usepackage[]{titletoc}
\usepackage{empheq}
\usepackage{titlesec}
\usepackage{mathpazo}
\usepackage{xfrac}
\usepackage{textcomp}
\usepackage{mathtools}
\usepackage{caption}
\usepackage{tabularray}
\usepackage{subcaption}
\usepackage[bottom]{footmisc}
\usepackage{pdfpages}
\usepackage{tabularx}
\usepackage{amsthm}
\usepackage[skins]{tcolorbox}
\titleformat{\chapter}[display]
  {\normalfont\bfseries}{}{0pt}{\Huge}
\usepackage{hyperref}
\newcommand{\hsp}{\hspace{20pt}}
\newcommand{\HRule}{\rule{\linewidth}{0.5mm}}
\newcommand{\R}{\mathbb{R}}
\newcommand{\C}{\mathbb{C}}
\theoremstyle{definition}
\newtheorem{thm}{Theorem}[chapter]
\newtheorem{definition}[thm]{Definition}
\newtheorem{lem}[thm]{Lemma}

% environment derived from framed.sty: see leftbar environment definition
\definecolor{formalshade}{rgb}{0.95,0.95,1}
\definecolor{darkblue}{rgb}{0.0, 0.0, 0.55}

\newenvironment{formal}{
  \def\FrameCommand{
    \hspace{1pt}
    {\color{darkblue}\vrule width 2pt}
    {\color{formalshade}\vrule width 4pt}
    \colorbox{formalshade}
  }
  \MakeFramed{\advance\hsize-\width\FrameRestore}
  \noindent\hspace{-4.55pt}% disable indenting first paragraph
  \begin{adjustwidth}{}{7pt}
  \vspace{2pt}\vspace{2pt}
}
{
  \vspace{2pt}\end{adjustwidth}\endMakeFramed
}

% allows multiple places to referrence the same footnote by using \footnote{\label{x}...} and \footnoteref{x}
\makeatletter
\newcommand\footnoteref[1]{\protected@xdef\@thefnmark{\ref{#1}}\@footnotemark}
\makeatother

\hbadness=100000
\begin{document}
\begin{titlepage}
	\begin{sffamily}
	\begin{center}
		\includegraphics[scale=0.3]{img/page_de_garde.jpg} \\[1cm]
		\HRule \\[0.4cm]
		{ \huge \bfseries LMECA2660 - Numerical Methods in Fluid Mechanics \\[0.4cm] }
	
		\HRule \\[1.5cm]
		\textsc{\LARGE Alexandre Or\'ekhoff \\ \LARGE Simon Desmidt}\\[3cm]
		{This summary may not be up-to-date, the newer version is available at this address: \hyperlink{https://github.com/SimonDesmidt/Syntheses}{https://github.com/SimonDesmidt/Syntheses}}
        \vfill
		\vspace{2cm}
		{\large Academic year 2025-2026 - Q2}
		\vspace{0.4cm}
		 
		\includegraphics[width=0.15\textwidth]{img/epl.png}
		
		UCLouvain\\
	
	\end{center}
	\end{sffamily}
\end{titlepage}

\setcounter{tocdepth}{1}
\tableofcontents
\chapter{Finite differences with uniform grid}
\section{Classical finite differences}
Let us define a function $u(\cdot)$ that depends on a variable $x$. Suppose that in the dimension $x$, we discretize the function uniformly with a step $h$ and the values at the nodes are written $u_i$. Then, by a Taylor development series,
\begin{equation}
	\begin{cases}
		u_{i+1} = u_i + h\left(\frac{\partial u}{\partial x}\right)_i + \frac{h^2}{2!}\left(\frac{\partial^2 u}{\partial x^2}\right)_i + \frac{h^3}{3!}\left(\frac{\partial^3 u}{\partial x^3}\right)_i + \frac{h^4}{4!} \left(\frac{\partial^4 u}{\partial x^4}\right)_i + \dots \\
		u_{i-1} = u_i - h\left(\frac{\partial u}{\partial x}\right)_i + \frac{h^2}{2!}\left(\frac{\partial^2 u}{\partial x^2}\right)_i - \frac{h^3}{3!}\left(\frac{\partial^3 u}{\partial x^3}\right)_i + \frac{h^4}{4!} \left(\frac{\partial^4 u}{\partial x^4}\right)_i - \dots \\
	\end{cases}
\end{equation}
This gives three possible finite-difference approximations:
\begin{equation}
	\begin{aligned}
		\left(\frac{\partial u}{\partial x}\right)_i = \frac{u_{i+1}-u_i}{h} + \mathcal{O}(h) & \qquad \text{(Forward differences)}\\
		\left(\frac{\partial u}{\partial x}\right)_i = \frac{u_{i}-u_{i-1}}{h} + \mathcal{O}(h) & \qquad \text{(Backward differences)}\\
		\left(\frac{\partial u}{\partial x}\right)_i = \frac{u_{i+1}-u_{i-1}}{2h} + \mathcal{O}(h^2) & \qquad \text{(Centered differences)}\\
	\end{aligned}
\end{equation}
This also gives, for the second order, 
\begin{equation}
	\left(\frac{\partial^2 u}{\partial x^2}\right)_i = \frac{u_{i+1}-2u_i+u_{i-1}}{h^2} - \frac{h^2}{12} \left(\frac{\partial^4u}{\partial x^4}\right)_i + \dots 
\end{equation}
\begin{itemize}
	\item [$\to$] Note: in general, discentered differences are only used for stability reasons.
\end{itemize}
\section{Richardson extrapolation}
Richardson extrapolation combines centered finite differences at different scales to get a better error:
\begin{equation}
	\begin{aligned}
		\textcolor{red}{\frac{4}{3}\Big [}&\left(\frac{\partial u}{\partial x}\right)_i = \frac{u_{i+1}-u_{i-1}}{2h} - \frac{h^2}{6}\left(\frac{\partial^3 u }{\partial x^3}\right)_i - \frac{h^4}{120}\left(\frac{\partial^5 u}{\partial x^5}\right)_i - \dots\textcolor{red}{\Big]}\\
		\textcolor{red}{\frac{-1}{3}\Big [}&\left(\frac{\partial u}{\partial x}\right)_i = \frac{u_{i+2}-u_{i-2}}{2(2h)} - \frac{(2h)^2}{6}\left(\frac{\partial^3 u }{\partial x^3}\right)_i - \frac{(2h)^4}{120}\left(\frac{\partial^5 u}{\partial x^5}\right)_i - \dots\textcolor{red}{\Big]}\\
		\Longrightarrow &\left(\frac{\partial u}{\partial x}\right)_i = \frac{8(u_{i+1}-u_{i-1})-(u_{i+2}-u_{i-2})}{12h} + \frac{h^4}{30}\left(\frac{\partial^5 u}{\partial x^5}\right)_i - \dots
	\end{aligned}
\end{equation}
With this method, the truncation error is of order $\mathcal{O}(h^4)$. In the same way, for second order,
\begin{equation}
	\left(\frac{\partial^2u}{\partial x^2}\right)_i = \frac{4}{3}\frac{u_{i+1}-2u_i+u_{i-1}}{h^2} - \frac{1}{3}\frac{u_{i+2}-2u_i+u_{i-2}}{(2h)^2} + \mathcal{O}(h^4)
\end{equation}
\section{Operators}
Let us define the following operators:
\begin{itemize}
	\item Forward difference: $\Delta u_i = u_{i+1} - u_i$;
	\item Backward difference: $\nabla u_i = u_{i}-u_{i-1}$;
	\item Centered difference: $\delta u_i = u_{i+1/2}-u_{i-1/2}$;
	\item Mean: $\mu u_i = \frac{1}{2}(u_{i+1/2}+u_{i-1/2})$;
	\item [$\to$] Note: $u_{i+1/2}$ and $u_{i-1/2}$ are not computable because they are not grid values, but can be used for derivations of other formulae.
	\item Identity operator: $Iu_i = u_i$;
	\item Forward operator: $Eu_i = u_{i+1}$;
	\item Backward operator: $E^{-1}u_i = u_{i-1}$;
	\item [$\to$] Note: $E^{-1} E = I$.
\end{itemize}
Those operators have the following properties:
\begin{itemize}
	\item $\mu \delta = \frac{1}{2}(E-E^{-1})$;
	\item $\mu^2 = I+\delta^2/4$;
\end{itemize}
The forward operator can be re-expressed using a Taylor development series:
\begin{equation}
	\begin{aligned}
		Eu_i &= u_{i+1} = u_i + h\frac{\partial }{\partial x} u_i + \frac{h^2}{2!} \frac{\partial^2}{\partial x^2} u_i + \frac{h^3}{3!} \frac{\partial^3}{\partial x^3}_i + \dots \\
		&= \left(I+hD+\frac{(hD)^2}{2!} + \frac{(hD)^3}{3!} + \dots\right)u_i = \exp(hD)u_i
	\end{aligned}
\end{equation}
From this, using a second Taylor development series,
\begin{equation}
	hD = \log(I+\Delta) = \Delta - \frac{\Delta^2}{2} + \frac{\Delta^3}{3} - \frac{\Delta^4}{4} + \dots 
\end{equation}
And, in the same way, 
\begin{equation}
	hD = -\log (I-\nabla) = \nabla + \frac{\nabla^2}{2} + \frac{\nabla^3}{3} + \frac{\nabla^4}{4} + \dots 
\end{equation}
We can do the same for another operator:
\begin{equation}
	\mu \delta = \frac{1}{2}(E-E^{-1}) = \frac{1}{2}\left(\exp(hD)-\exp(-hD)\right) = \sinh(hD)
\end{equation}
and we can use the Taylor series for $arc\sinh(x)$ but it is not very useful. By the property that $\mu^2 = I+\delta^2/4$, we get another form:
\begin{equation}
	hD = \mu\delta \left(I-\frac{1}{6}\delta^2 + \frac{1}{30}\delta^4 - \frac{140}{\delta^6} + \dots\right)
\end{equation}
If we keep only the first order term, we find the centered-difference scheme, and the terms up to second order give the Richardson extrapolation. 
\begin{itemize}
	\item [$\to$] Note: in any scheme, using more information (more values, e.g. $u_{i+2}, u_{i+3}, \dots$) gives a more accurate solution and the order of the truncation error increases (e.g. to $\mathcal{O}(h^3)$). 
\end{itemize}
page 8.
\end{document}