\documentclass[12pt, openany]{report}
\usepackage[utf8]{inputenc}
\usepackage[T1]{fontenc}
\usepackage{amsmath,amsfonts,amssymb}
\usepackage{amssymb}
\usepackage{multicol}
\usepackage[a4paper,left=2.5cm,right=2.5cm,top=2.5cm,bottom=2.5cm]{geometry}
\usepackage[english]{babel}
\usepackage{libertine}
\usepackage{graphicx}
\usepackage{wrapfig}
\usepackage{algorithm}
\usepackage{algpseudocode}
\usepackage{float}
\usepackage{enumitem}
\usepackage{pythonhighlight}
\usepackage[]{titletoc}
\usepackage{empheq}
\usepackage{titlesec}
\usepackage{mathpazo}
\usepackage{xfrac}
\usepackage{textcomp}
\usepackage{mathtools}
\usepackage{caption}
\usepackage{tabularray}
\usepackage{subcaption}
\usepackage[bottom]{footmisc}
\usepackage{pdfpages}
\usepackage{tabularx}
\usepackage{amsthm}
\usepackage[skins]{tcolorbox}
\titleformat{\chapter}[display]
  {\normalfont\bfseries}{}{0pt}{\Huge}
\usepackage{hyperref}
\newcommand{\hsp}{\hspace{20pt}}
\newcommand{\HRule}{\rule{\linewidth}{0.5mm}}
\newcommand{\R}{\mathbb{R}}
\newcommand{\C}{\mathbb{C}}
\newcommand{\Z}{\mathbb{Z}}
\theoremstyle{definition}
\newtheorem{thm}{Theorem}[chapter]
\newtheorem{definition}[thm]{Definition}
\newtheorem{lem}[thm]{Lemma}

% environment derived from framed.sty: see leftbar environment definition
\definecolor{formalshade}{rgb}{0.95,0.95,1}
\definecolor{darkblue}{rgb}{0.0, 0.0, 0.55}

\newenvironment{formal}{
  \def\FrameCommand{
    \hspace{1pt}
    {\color{darkblue}\vrule width 2pt}
    {\color{formalshade}\vrule width 4pt}
    \colorbox{formalshade}
  }
  \MakeFramed{\advance\hsize-\width\FrameRestore}
  \noindent\hspace{-4.55pt}% disable indenting first paragraph
  \begin{adjustwidth}{}{7pt}
  \vspace{2pt}\vspace{2pt}
}
{
  \vspace{2pt}\end{adjustwidth}\endMakeFramed
}

% allows multiple places to referrence the same footnote by using \footnote{\label{x}...} and \footnoteref{x}
\makeatletter
\newcommand\footnoteref[1]{\protected@xdef\@thefnmark{\ref{#1}}\@footnotemark}
\makeatother

\hbadness=100000
\begin{document}
\begin{titlepage}
	\begin{sffamily}
	\begin{center}
		\includegraphics[scale=0.3]{img/page_de_garde.jpg} \\[1cm]
		\HRule \\[0.4cm]
		{ \huge \bfseries LMECA2660 - Numerical Methods in Fluid Mechanics \\[0.4cm] }
	
		\HRule \\[1.5cm]
		\textsc{\LARGE Alexandre Or\'ekhoff \\ \LARGE Simon Desmidt}\\[3cm]
		{This summary may not be up-to-date, the newer version is available at this address: \hyperlink{https://github.com/SimonDesmidt/Syntheses}{https://github.com/SimonDesmidt/Syntheses}}
        \vfill
		\vspace{2cm}
		{\large Academic year 2025-2026 - Q2}
		\vspace{0.4cm}
		 
		\includegraphics[width=0.15\textwidth]{img/epl.png}
		
		UCLouvain\\
	
	\end{center}
	\end{sffamily}
\end{titlepage}

\setcounter{tocdepth}{1}
\tableofcontents
\chapter{Finite differences with uniform grid}
\section{Classical finite differences}
Let us define a function $u(\cdot)$ that depends on a variable $x$. Suppose that in the dimension $x$, we discretize the function uniformly with a step $h$ and the values at the nodes are written $u_i$. Then, by a Taylor development series,
\begin{equation}
	\begin{cases}
		u_{i+1} = u_i + h\left(\frac{\partial u}{\partial x}\right)_i + \frac{h^2}{2!}\left(\frac{\partial^2 u}{\partial x^2}\right)_i + \frac{h^3}{3!}\left(\frac{\partial^3 u}{\partial x^3}\right)_i + \frac{h^4}{4!} \left(\frac{\partial^4 u}{\partial x^4}\right)_i + \dots \\
		u_{i-1} = u_i - h\left(\frac{\partial u}{\partial x}\right)_i + \frac{h^2}{2!}\left(\frac{\partial^2 u}{\partial x^2}\right)_i - \frac{h^3}{3!}\left(\frac{\partial^3 u}{\partial x^3}\right)_i + \frac{h^4}{4!} \left(\frac{\partial^4 u}{\partial x^4}\right)_i - \dots \\
	\end{cases}
\end{equation}
This gives three possible finite-difference approximations:
\begin{equation}
	\begin{aligned}
		\left(\frac{\partial u}{\partial x}\right)_i = \frac{u_{i+1}-u_i}{h} + \mathcal{O}(h) & \qquad \text{(Forward differences)}\\
		\left(\frac{\partial u}{\partial x}\right)_i = \frac{u_{i}-u_{i-1}}{h} + \mathcal{O}(h) & \qquad \text{(Backward differences)}\\
		\left(\frac{\partial u}{\partial x}\right)_i = \frac{u_{i+1}-u_{i-1}}{2h} + \mathcal{O}(h^2) & \qquad \text{(Centered differences)}\\
	\end{aligned}
\end{equation}
This also gives, for the second order, 
\begin{equation}
	\left(\frac{\partial^2 u}{\partial x^2}\right)_i = \frac{u_{i+1}-2u_i+u_{i-1}}{h^2} - \frac{h^2}{12} \left(\frac{\partial^4u}{\partial x^4}\right)_i + \dots 
\end{equation}
\begin{itemize}
	\item [$\to$] Note: in general, discentered differences are only used for stability reasons.
\end{itemize}
\section{Richardson extrapolation}
Richardson extrapolation combines centered finite differences at different scales to get a better error:
\begin{equation}
	\begin{aligned}
		\textcolor{red}{\frac{4}{3}\Big [}&\left(\frac{\partial u}{\partial x}\right)_i = \frac{u_{i+1}-u_{i-1}}{2h} - \frac{h^2}{6}\left(\frac{\partial^3 u }{\partial x^3}\right)_i - \frac{h^4}{120}\left(\frac{\partial^5 u}{\partial x^5}\right)_i - \dots\textcolor{red}{\Big]}\\
		\textcolor{red}{\frac{-1}{3}\Big [}&\left(\frac{\partial u}{\partial x}\right)_i = \frac{u_{i+2}-u_{i-2}}{2(2h)} - \frac{(2h)^2}{6}\left(\frac{\partial^3 u }{\partial x^3}\right)_i - \frac{(2h)^4}{120}\left(\frac{\partial^5 u}{\partial x^5}\right)_i - \dots\textcolor{red}{\Big]}\\
		\Longrightarrow &\left(\frac{\partial u}{\partial x}\right)_i = \frac{8(u_{i+1}-u_{i-1})-(u_{i+2}-u_{i-2})}{12h} + \frac{h^4}{30}\left(\frac{\partial^5 u}{\partial x^5}\right)_i - \dots
	\end{aligned}
\end{equation}
With this method, the truncation error is of order $\mathcal{O}(h^4)$. In the same way, for second order,
\begin{equation}
	\left(\frac{\partial^2u}{\partial x^2}\right)_i = \frac{4}{3}\frac{u_{i+1}-2u_i+u_{i-1}}{h^2} - \frac{1}{3}\frac{u_{i+2}-2u_i+u_{i-2}}{(2h)^2} + \mathcal{O}(h^4)
\end{equation}
\section{Operators}
Let us define the following operators:
\begin{itemize}
	\item Forward difference: $\Delta u_i = u_{i+1} - u_i$;
	\item Backward difference: $\nabla u_i = u_{i}-u_{i-1}$;
	\item Centered difference: $\delta u_i = u_{i+1/2}-u_{i-1/2}$;
	\item Mean: $\mu u_i = \frac{1}{2}(u_{i+1/2}+u_{i-1/2})$;
	\item [$\to$] Note: $u_{i+1/2}$ and $u_{i-1/2}$ are not computable because they are not grid values, but can be used for derivations of other formulae.
	\item Identity operator: $Iu_i = u_i$;
	\item Forward operator: $Eu_i = u_{i+1}$;
	\item Backward operator: $E^{-1}u_i = u_{i-1}$;
	\item [$\to$] Note: $E^{-1} E = I$.
\end{itemize}
Those operators have the following properties:
\begin{itemize}
	\item $\mu \delta = \frac{1}{2}(E-E^{-1})$;
	\item $\mu^2 = I+\delta^2/4$;
\end{itemize}
The forward operator can be re-expressed using a Taylor development series:
\begin{equation}
	\begin{aligned}
		Eu_i &= u_{i+1} = u_i + h\frac{\partial }{\partial x} u_i + \frac{h^2}{2!} \frac{\partial^2}{\partial x^2} u_i + \frac{h^3}{3!} \frac{\partial^3}{\partial x^3}_i + \dots \\
		&= \left(I+hD+\frac{(hD)^2}{2!} + \frac{(hD)^3}{3!} + \dots\right)u_i = \exp(hD)u_i
	\end{aligned}
\end{equation}
From this, using a second Taylor development series,
\begin{equation}
	hD = \log(I+\Delta) = \Delta - \frac{\Delta^2}{2} + \frac{\Delta^3}{3} - \frac{\Delta^4}{4} + \dots 
\end{equation}
And, in the same way, 
\begin{equation}
	hD = -\log (I-\nabla) = \nabla + \frac{\nabla^2}{2} + \frac{\nabla^3}{3} + \frac{\nabla^4}{4} + \dots 
\end{equation}
We can do the same for another operator:
\begin{equation}
	\mu \delta = \frac{1}{2}(E-E^{-1}) = \frac{1}{2}\left(\exp(hD)-\exp(-hD)\right) = \sinh(hD)
\end{equation}
and we can use the Taylor series for $arc\sinh(x)$ but it is not very useful. By the property that $\mu^2 = I+\delta^2/4$, we get another form:
\begin{equation}
	hD = \mu\delta \left(I-\frac{1}{6}\delta^2 + \frac{1}{30}\delta^4 - \frac{140}{\delta^6} + \dots\right)
\end{equation}
If we keep only the first order term, we find the centered-difference scheme, and the terms up to second order give the Richardson extrapolation. 
\begin{itemize}
	\item [$\to$] Note: in any scheme, using more information (more values, e.g. $u_{i+2}, u_{i+3}, \dots$) gives a more accurate solution and the order of the truncation error increases (e.g. to $\mathcal{O}(h^3)$). 
\end{itemize}
\section{2D Laplacian}
For finite differences in 2D, we can define several types of stencils. For a second-order error, there is the cross operator, which is simply the sum of classical centered finite differences on both axes, and the box operator. This operator is a linear combination of the cross operator using the medians of the square, and the one that uses the diagonals of the square (see \ref{fig:cross-op}).
\begin{figure}[H]
	\centering
	\includegraphics[width=.7\textwidth]{img/cross_op.png}
	\caption{Cross operator and box operator.}
	\label{fig:cross-op}
\end{figure}
The box operator expresses the following quantity:
\begin{equation}
	h^2 \nabla^2 \left(u+\frac{h^2}{12}\nabla^2 u  + \dots \right) = h^2 \left(\nabla^2 u + \frac{h^2}{12}\nabla^2 (\nabla^2 u)\right)
\end{equation}
Those stencils can be generalized to higher orders using more points (bigger cross and bigger square). 
\begin{itemize}
	\item [$\to$] Note: the coefficients are found using the constraint that the truncation error is independent of orientation of the stencil. 
\end{itemize}
\section{Convection equation}
The convection equation is 
\begin{equation}
	\frac{\partial u}{\partial t} + c \frac{\partial u}{\partial x}=0
\end{equation}
The analytic solution of this equation on an infinite domain, for a speed $c$ constant, is 
\begin{equation}
	u(x,t) = A(t)e^{ikx} = A(0)e^{ik(x-ct)}
\end{equation}
On a periodic domain of length $L$, the solution is 
\begin{equation}
	\begin{aligned}
		u(x,t) = \sum_{k=-\infty}^\infty A_k(t)e^{ikx}\qquad k=\frac{2\pi}{L}p \qquad p\in \Z\\
		\Longrightarrow u(x,t) = \sum_{p=-\infty}^\infty A_p(t) e^{i\frac{2\pi x}{L}}
	\end{aligned}
\end{equation}
where the coefficients verify the condition $A_k = A_{-k}^*$ for all $k$, since the solution must be real. In the exact solution, all the modes have the same speed $c$. However, it is not the case when we use explicit finite differences. Let us show it in the case of an infinite domain (same thing happens for a periodic domain, adding the sum on $p$):\\
Let $u_i(t) = A(t) e^{j kx_i}$. Then,
\begin{equation}
	\begin{aligned}
		\left. \frac{\partial u}{\partial x}\right|_i = A j k^* \exp(jkx_i) \Longrightarrow \frac{dA}{dt} + j k^* c A = 0 \\ \Longrightarrow u_i(t) = A(0) e^{j(kx_i - k^* ct)} = A(0) e^{ik\left(x_i - \frac{k^*h}{kh}ct\right)}
	\end{aligned}
\end{equation}
we call $k^*$ the modified wave number. It is different from $k$ because all the modes do not move at the same speed. For example, for the E2 stencil, its expression is derived in the following way:
\begin{equation}
	\begin{aligned}
		\left. \frac{\partial u}{\partial x}\right|_i = \frac{u_{i+1}-u_i}{2h} = \frac{A}{h} \frac{1}{2}(e^{jkh}-e^{-jkh})e^{jkx_i} = \frac{A}{h} j \sin(kh)e^{ikx_i} = A j k^* e^{jkx_i}\\
		\Longrightarrow k^*h = \sin(kh)
	\end{aligned}
\end{equation}
This is a \textbf{phase} error, as the modes do not \textbf{move} with the right \textbf{velocity}.\\
By a Taylor development, 
\begin{equation}
	\frac{k^*h}{kh} = 1 - \frac{(kh)^2}{6} + \mathcal{O}((kh)^4)
\end{equation}
and although $k$ is not constant for all modes, the error of the stencil is still of the same order: $\mathcal{O}((kh)^2)$. Moreover, the speed of the modes is $c^* = \frac{k^*h}{kh}c$.
\begin{figure}[H]
	\centering 
	\includegraphics[width=.5\textwidth]{img/convection-kmod.png}
	\caption{Evolution of the exact wave number and the modified wave number.}
	\label{fig:kh/kh}
\end{figure}
As the error increases when $k$ increases, it is important to use a very refined grid so that all points whose amplitude is non negligible have $k^* h \approx kh$. 
\section{Implicit finite differences}
The general scheme of implicit finite differences is the following:
\begin{equation}\label{eq:I-order1}
	\begin{aligned}	
		\frac{\partial u}{\partial x}_i + \alpha\frac{1}{2}\left(\left.\frac{\partial u}{\partial x}\right|_{i+1}+\left.\frac{\partial u}{\partial x}\right|_{i-1}\right) + \beta \frac{1}{2}\left(\left.\frac{\partial u}{\partial x}\right|_{i+2} + \left.\frac{\partial u}{\partial x}\right|_{i-2}\right) \\
		= a \frac{u_{i+1}-u_{i-1}}{2h} + b\frac{u_{i+2}+u_{i-2}}{4h} + c \frac{u_{i+3}-u_{i-3}}{6h}
	\end{aligned}
\end{equation}
Usually, we use $\beta=c=0$ to keep only the nearest grid points to $i$. Those schemes are called compact. \\
We can use $\left.\frac{\partial u}{\partial x}\right|_{i} = A j k^* e^{jkx_i}$ and $\frac{u_{i+1}-u_{i-1}}{2h} = \sin(kh)$ to show that 
\begin{equation}
	k^*h = \frac{a\sin(kh) + \frac{b}{2}\sin(2kh) + \frac{c}{3}\sin(3kh)}{1+\alpha \cos(kh)+\beta \cos(2kh)}
\end{equation}
We can do a Taylor development of this quantity to get different schemes:
\begin{equation}
	\begin{aligned}
		\text{Taylor of order 1:}\qquad  1+\alpha+\beta =& a+b+c \Longrightarrow \text{Error of order 2}\\
		\text{Taylor of order 2:}\qquad  3(\alpha+2^2\beta) =& a+2^2b+3^2c \Longrightarrow \text{Error of order 4}\\
		\text{Taylor of order 3:}\qquad  5(\alpha+2^4\beta) =& a+2^4b+3^4c \Longrightarrow \text{Error of order 6}\\
		\text{Taylor of order 4:}\qquad  7(\alpha+2^6\beta) =& a+2^6b+3^6c \Longrightarrow \text{Error of order 8}\\
		\text{Taylor of order 5:}\qquad  9(\alpha+2^8\beta) =& a+2^8b+3^8c \Longrightarrow \text{Error of order 10}\\
	\end{aligned}
\end{equation}
For example, in the case where $\beta= 0$ and $\alpha \neq 0$, the system is tridiagonal and the solver has a time complexity of $\mathcal{O}(N)$. For $\beta=0$ and $\alpha=0$, we find the explicit scheme. 
\begin{itemize}
	\item [$\to$] Note: The I4o scheme is the scheme with the smallest constant before the $(kh)^4$ using all parameters. For this one, the sign of that constant is positive, meaning that it goes above the correct velocity. This behaviour is not present on the other scheme. 
\end{itemize}
\section{Diffusion equation}
The diffusion equation is 
\begin{equation}\label{eq:diffusion}
	\frac{\partial u}{\partial t} = \alpha \frac{\partial^2 u}{\partial x^2}
\end{equation}
This equation is not reversible because it represents a loss of information: the high-wave-number terms decay fast and have no impact after a small time.\\
The exact solution is $u(x,t) = A_k(t)e^{jkx}$ with $A_k(t) = A_k(0)e^{\alpha k^2 t}$. 
\subsection{Explicit scheme}
The explicit finite differences scheme is 
\begin{equation}
	\left.\frac{\partial^2 u}{\partial x^2}\right|_i = a\frac{u_{i+1}-2u_i+u_{i-1}}{h^2} + b\frac{u_{i+2}-2u_i+u_{i-2}}{4h^2} + c\frac{u_{i+3}-2u_i+u_{i-3}}{9h^2}
\end{equation}
Let $u_i(t) = A(t) e^{jkx_i}$. Then, 
\begin{equation}
	\frac{dA}{dt} = -\alpha (k^2)^* A \Longrightarrow A(t) = A(0)e^{-\alpha (k^2)^* t} e^{jkx_i} \Longrightarrow u_i(t) = A(0)e^{-\alpha k^2 \left(\frac{\left(k^2\right)^*h^2}{k^2 h^2}\right)}e^{jkx_i}
\end{equation}
This is a \textbf{amplitude} error, as the mode do not decay with the proper \textbf{rate}. \\
For example, for the explicit scheme of order 2 ($a=1$, $b=c=0$), the modified $k^2$ is given by
\begin{equation}
	(k^1)^*h^2 = 4\sin^2 \left(\frac{kh}{2}\right) \Longleftrightarrow \frac{(k^2)^* h^2}{k^2 h^2} = \frac{\sin^2 \left(\frac{kh}{2}\right)}{\left(\frac{kh}{2}\right)^2}
\end{equation}
\subsection{Implicit schemes}
The general form of the implicit scheme follows the same reasoning as for \eqref{eq:I-order1}.
\begin{equation}
	\begin{aligned}
		\left.\frac{\partial^2 u}{\partial x^2}\right|_i + \alpha \frac{1}{2}\left(\left.\frac{\partial^2 u}{\partial x^2}\right|_{i+1} + \left.\frac{\partial^2 u}{\partial x^2}\right|_{i-1}\right) + \beta \frac{1}{2}\left(\left.\frac{\partial^2 u}{\partial x^2}\right|_{i+2} + \left.\frac{\partial^2 u}{\partial x^2}\right|_{i-2}\right) \\
		= a \frac{u_{i+1}-2u_i+u_{i-1}}{h^2} + b \frac{u_{i+2}-2u_i+u_{i-2}}{4h^2} + c \frac{u_{i+3}-2u_i+u_{i-3}}{9h^2}
	\end{aligned}
\end{equation}
By using $\left.\frac{\partial^2 u}{\partial x^2}\right|_i = -A(t) (k^2)^*e^{jkx_i}$, we can show that 
\begin{equation}
	(k^2)^*h^2 = \frac{4\left[a\sin^2\left(\frac{kh}{2}\right) + \frac{b}{4}\sin^2 \left(\frac{2kh}{2}\right) + \frac{c}{9}\sin^2 \left(\frac{3kh}{2}\right)\right]}{1+\alpha \cos(kh)+\beta\cos(2kh)}
\end{equation}
As previously, we can do the Taylor development series of this expression to get different schemes:
\begin{equation}
	\begin{aligned}
		\text{Taylor of order 1:}\qquad  1+\alpha+\beta =& a+b+c \Longrightarrow \text{Error of order 2}\\
		\text{Taylor of order 2:}\qquad  2\cdot3(\alpha+2^2\beta) =& a+2^2b+2^2c \Longrightarrow \text{Error of order 4}\\
		\text{Taylor of order 3:}\qquad  3\cdot5(\alpha+2^4\beta) =& a+2^4b+2^4c \Longrightarrow \text{Error of order 6}\\
		\text{Taylor of order 4:}\qquad  4\cdot7(\alpha+2^6\beta) =& a+2^6b+2^6c \Longrightarrow \text{Error of order 8}\\
		\text{Taylor of order 5:}\qquad  5\cdot9(\alpha+2^8\beta) =& a+2^8b+2^8c \Longrightarrow \text{Error of order 10}\\
	\end{aligned}
\end{equation}
\begin{figure}[H]
	\centering 
	\includegraphics[width=.5\textwidth]{img/diffusion-kmod.png}
	\caption{Evolution of the modified $k^2$ with $k^2$.}
	\label{fig:diffusion-kmod}
\end{figure}

\chapter{Fourier document}
\end{document}