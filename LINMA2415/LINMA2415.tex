\documentclass[12pt, openany]{report}
\usepackage[utf8]{inputenc}
\usepackage[T1]{fontenc}
\usepackage{amsmath,amsfonts,amssymb}
\usepackage{amssymb}
\usepackage{multicol}
\usepackage[a4paper,left=2.5cm,right=2.5cm,top=2.5cm,bottom=2.5cm]{geometry}
\usepackage[english]{babel}
\usepackage{libertine}
\usepackage{graphicx}
\usepackage{wrapfig}
\usepackage{algorithm}
\usepackage{algpseudocode}
\usepackage{float}
\usepackage{enumitem}
\usepackage{pythonhighlight}
\usepackage[]{titletoc}
\usepackage{empheq}
\usepackage{titlesec}
\usepackage{mathpazo}
\usepackage{xfrac}
\usepackage{textcomp}
\usepackage{mathtools}
\usepackage{caption}
\usepackage{tabularray}
\usepackage{subcaption}
\usepackage[bottom]{footmisc}
\usepackage{pdfpages}
\usepackage{tabularx}
\usepackage{amsthm}
\usepackage{listings}
\usepackage[skins]{tcolorbox}
\titleformat{\chapter}[display]
  {\normalfont\bfseries}{}{0pt}{\Huge}
\usepackage{hyperref}
\newcommand{\hsp}{\hspace{20pt}}
\newcommand{\HRule}{\rule{\linewidth}{0.5mm}}
\newcommand{\R}{\mathbb{R}}
\newcommand{\C}{\mathbb{C}}
\theoremstyle{definition}
\newtheorem{thm}{Theorem}[chapter]
\newtheorem{definition}[thm]{Definition}
\newtheorem{lem}[thm]{Lemma}


\definecolor{mGreen}{rgb}{0,0.6,0}
\definecolor{mGray}{rgb}{0.5,0.5,0.5}
\definecolor{mPurple}{rgb}{0.58,0,0.82}
\definecolor{backgroundColour}{rgb}{0.95,0.95,0.92}
\definecolor{light-gray}{gray}{0.95}
\newcommand{\code}[1]{\colorbox{light-gray}{\texttt{#1}}}

\lstdefinestyle{CppStyle}{
    backgroundcolor=\color{backgroundColour},   
    commentstyle=\color{mGreen},
    keywordstyle=\color{magenta},
    numberstyle=\tiny\color{mGray},
    stringstyle=\color{mPurple},
    basicstyle=\footnotesize,
    breakatwhitespace=false,         
    breaklines=true,                 
    captionpos=b,                    
    keepspaces=true,                 
    numbers=left,                    
    numbersep=5pt,                  
    showspaces=false,                
    showstringspaces=false,
    showtabs=false,                  
    tabsize=2,
    language=C
}

\hbadness=100000
\begin{document}
\begin{titlepage}
    \begin{sffamily}
    \begin{center}
        \includegraphics[scale=0.25]{img/page_de_garde.png} \\[1cm]
        \HRule \\[0.4cm]
        { \huge \bfseries LINMA2415 Quantitative Energy Economics \\[0.4cm] }
    
        \HRule \\[1.5cm]
        \textsc{\LARGE Issambre L'Hermite Dumont}\\[3cm]
        {This summary may not be up-to-date, the newer version is available at this address: \hyperlink{https://github.com/SimonDesmidt/Syntheses}{https://github.com/SimonDesmidt/Syntheses}}
        \vfill
        \vspace{2cm}
        {\large Academic year 2025-2026 - Q2}
        \vspace{0.4cm}
         
        \includegraphics[width=0.15\textwidth]{img/epl.png}
        
        UCLouvain\\
    
    \end{center}
    \end{sffamily}
\end{titlepage}

\setcounter{tocdepth}{1}
\tableofcontents
\chapter{Reminder about optimization}
\section{Formulations}
\begin{definition}
    The primal problem is the original optimization problem that we want to solve. It is formulated as:
    \begin{equation}
        \begin{aligned}
            \max_{x} \quad & c^T x \\
            \text{s.t.} \quad & A x \leq b \\
            & x \geq 0
        \end{aligned}
    \end{equation}
\end{definition}
\begin{definition}
    The dual problem is derived from the primal problem. It is formulated as:
    \begin{equation}
        \begin{aligned}
            \min_{y} \quad & b^T y \\
            \text{s.t.} \quad & A^T y \geq c \\
            & y \geq 0
        \end{aligned}
    \end{equation}
\end{definition}
Another way to look at the dual problem is by using the Lagrangian function. Consider now the primal under the form:
\begin{equation}
    \begin{aligned}
        p^* = \min \quad & f_0(x)\\
        \text{s.t.} \quad &f_i(x) \leq 0, \quad i = 1, \ldots, m \\
        & h_j(x) = 0, \quad  j = 1, \ldots, p
    \end{aligned}
\end{equation}
\begin{definition}
    The Lagrangian function is defined as:
    \begin{equation}
        L(x,\lambda, \nu) = f_0(x) + \sum_{i=1}^m \lambda_i f_i(x) + \sum_{j=1}^p \nu_j h_j(x)
    \end{equation}
\end{definition}
\begin{definition}
    The Lagrange dual function is defined as:
    \begin{equation}
        g(\lambda, \nu) = \inf_x L(x, \lambda, \nu)
    \end{equation}
\end{definition}
\begin{definition}
    The Lagrange dual problem is defined as:
    \begin{equation}
        \begin{aligned}
            d^* = \max \quad & g(\lambda, \nu) \\
            \text{s.t.} \quad & \lambda \geq 0
        \end{aligned}
    \end{equation}
\end{definition}
\section{Optimality conditions}
\begin{definition}
    Weak duality is when $d^* \leq p^*$, it always holds and can be used to find bounds on the optimal value of the primal problem.
\end{definition}
\begin{definition}
    Strong duality is when $d^* = p^*$, does not holds in general, but usually holds for convex problems.
\end{definition}
\begin{definition}
    Complementary slackness is a condition that holds at optimality, it states that for each constraint, either the constraint is active (i.e. $f_i(x) = 0$) or the corresponding dual variable is zero (i.e. $\lambda_i = 0$).
    \begin{equation}
        0 \leq \lambda_i^* \perp f_i(x^*) \leq 0
    \end{equation}
\end{definition}
\begin{definition}
    The KKT conditions are a set of necessary conditions for optimality. Consider this problem:
    \begin{equation}
        \begin{aligned}
            \max &f(x,y) \\
            \text{s.t.} & Ax + By \leq b \qquad (\lambda) \\
            & Cx + Dy = d \qquad (\mu) \\
            & x \geq 0 \qquad (\lambda_2)
        \end{aligned}
    \end{equation}
    The KKT conditions are:
    \begin{equation}
        \begin{aligned}
            Cx + & Dy - d = 0 \\
            0 \leq \lambda \perp & Ax + By -b \leq 0 \\
            0 \leq x \perp & \lambda^T A + \mu^T C - \nabla_x f(x,y)^T \geq 0 \\
            \lambda^T B + &  \mu^T D - \nabla_y f(x,y)^T = 0
        \end{aligned}
    \end{equation}
    And the Lagrangian function can be written as:
    \begin{equation}
        L(x,y,\lambda, \mu) = f(x,y) + \lambda^T (b - Ax - By) + \mu^T (d - Cx - Dy) + \lambda_2^Tx
    \end{equation}
\end{definition}
\end{document}