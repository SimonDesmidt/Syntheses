\documentclass[12pt, openany]{report}
\usepackage[utf8]{inputenc}
\usepackage[T1]{fontenc}
\usepackage{amsmath,amsfonts,amssymb}
\usepackage{amssymb}
\usepackage{multicol}
\usepackage[a4paper,left=2.5cm,right=2.5cm,top=2.5cm,bottom=2.5cm]{geometry}
\usepackage[english]{babel}
\usepackage{libertine}
\usepackage{graphicx}
\usepackage{wrapfig}
\usepackage{algorithm}
\usepackage{algpseudocode}
\usepackage{float}
\usepackage{enumitem}
\usepackage{pythonhighlight}
\usepackage[]{titletoc}
\usepackage{empheq}
\usepackage{titlesec}
\usepackage{mathpazo}
\usepackage{xfrac}
\usepackage{textcomp}
\usepackage{mathtools}
\usepackage{caption}
\usepackage{tabularray}
\usepackage{subcaption}
\usepackage[bottom]{footmisc}
\usepackage{pdfpages}
\usepackage{tabularx}
\usepackage{amsthm}
\usepackage[skins]{tcolorbox}
\titleformat{\chapter}[display]
  {\normalfont\bfseries}{}{0pt}{\Huge}
\usepackage{hyperref}
\newcommand{\hsp}{\hspace{20pt}}
\newcommand{\HRule}{\rule{\linewidth}{0.5mm}}
\newcommand{\R}{\mathbb{R}}
\newcommand{\C}{\mathbb{C}}
\theoremstyle{definition}
\newtheorem{thm}{Theorem}[chapter]
\newtheorem{definition}[thm]{Definition}
\newtheorem{lem}[thm]{Lemma}

\hbadness=100000
\begin{document}
\begin{titlepage}
    \begin{sffamily}
    \begin{center}
        \includegraphics[scale=2.5]{img/page_de_garde.png} \\[1cm]
        \HRule \\[0.4cm]
        { \huge \bfseries LMECA2300 Advanced Numerical Methods \\[0.4cm] }
    
        \HRule \\[1.5cm]
        \textsc{\LARGE Simon Desmidt}\\[1cm]
        \vfill
        \vspace{2cm}
        {\large Academic year 2024-2025 - Q2}
        \vspace{0.4cm}
         
        \includegraphics[width=0.15\textwidth]{img/epl.png}
        
        UCLouvain\\
    
    \end{center}
    \end{sffamily}
\end{titlepage}

\setcounter{tocdepth}{1}
\tableofcontents
\chapter{2-D acoustic and electromagnetic waves}
\section{Physical laws}
The expression of the force of the source applied to its surrounding is derived from Newton's law $F=ma$:
\begin{equation}
  \rho_0 \frac{\partial \vec{u}}{\partial t} + \nabla p = r_v
\end{equation}
where $\rho_0$ is the average density of the surrounding, $\vec{u}$ is the velocity field, $p$ is the variation of pressure and $r_v$ is the "pressure force". \\
The conservation law of energy is 
\begin{equation}
  \nabla \cdot \vec{u} + \chi \frac{\partial p}{\partial t}=s_v
\end{equation}
where $\chi$ is the compressibility $[kg^{-1}ms^2]$ and is given by the equation $\frac{\rho}{\rho_0}=\chi p$. $s_v$ is the "velocity source".\\
\section{Wave equation}
The wave equation is
\begin{equation}
	\nabla^2p-\rho_0\chi \frac{\partial^2 p}{\partial t^2} - -\rho_0 \frac{\partial s_V}{\partial t}
\end{equation}
In 1D, with $s_V=0$, the solution is any function $f\left(t-\frac{x}{v}\right)$ or $g\left(x-tv\right)$ with $v=1/\sqrt{\rho_0\chi}$. 
\section{Plane wave}
In the plane, the standard wave is given by 
\begin{equation}
	p(x,t)=p_0 \cos(\omega t-kx)
\end{equation}
where the phase velocity is $v_{ph} = \frac{\omega}{k}=\frac{1}{\sqrt{\chi \rho_0}}$.\\
The wave impedance is $\nu = \frac{p}{u_x} = \sqrt{\frac{\rho_0}{\chi}} \ [kgm^{-2}s^{-1}]$.
\end{document}